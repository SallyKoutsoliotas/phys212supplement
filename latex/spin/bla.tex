\begin{problem}
Consider the Stern-Gerlach (SG) experiments described below.
\begin{itemize}
\item[(a)] A beam of electrons, prepared in a certain state $\ket{\psi}$,
is emitted from an electron source. The beam passes into a Stern-Gerlach (SG)
device which measures the $z$-component of spin $S_z$ as shown below. 
A total of 1000 electrons go through this apparatus, of which 
exactly 400 are measured to have $S_z=+\hbar/2$ 
(spin-up along the $z$-axis) and exactly 600 are measured to have 
$S_z=-\hbar/2$ (spin-down along the $z$-axis). 
\begin{center}
\includegraphics[height=0.2\textwidth]{spin/SG_z.eps}   
\end{center}
Write a possible normalized state $\ket{\psi}$ for the electrons in 
the initial beam that corresponds to the observations in this experiment.

\item[(b)] 

You do the same experiment again with the same apparatus.  Again, you 
send 1000 electrons through the system, but this time you find 
382 electrons with $S_z = +\hbar/2$ and 618 electrons with $S_z = -\hbar/2$.
Should you conclude that the electrons in the initial beam have a
state that is different than the one from part (a)? Explain why or why not.

\end{itemize}

\end{problem}
