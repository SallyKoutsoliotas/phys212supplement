\newpage

\section*{Problems}
\markright{PROBLEMS}

\begin{problem}
Write either a poem, a song, or a few sentences explaining
what it is about spin -- on a quantum level -- that can't be
explained by classical laws of physics. \label{prob:spin_poem}
\end{problem}


\begin{problem}
Suppose an electron is known to have a $z$-component of spin $S_z$ of $+\hbar/2$; that is, we know it is in the $|\mbox{$+z$}\rangle$ state.
\begin{enumerate} 
\item   Find the probability that a measurement of
    its spin along the $x$-axis gives the value $+\hbar/2$.
\item Find the probability that a measurement of its spin along
    the $y$-axis gives the value $-\hbar/2$.
 \end{enumerate}
\label{prob:spin_i}
\end{problem}

\begin{problem}
An electron is known to be in the spin state $|\psi\rangle =
  \frac{3}{5}|\mbox{$+z$}\rangle + \frac{4}{5}|\mbox{$-z$}\rangle$.

  \begin{enumerate}
  \item The electron is sent
  through a device that measures its spin angular momentum along the
  $x$-direction.  Compute the probability of obtaining the result
  $+\hbar/2$ for the $x$-component of spin.  Compute the
  probability of obtaining $-\hbar/2$ for this measurement.  (Check to make
  sure that your probabilities add up to 1.)
  \item The electron initially in the state $|\psi\rangle$ specified
  above is sent through a device that measures its spin angular
  momentum along the $y$-direction.  Compute the probability of
  obtaining the result $+\hbar/2$ for the $y$-component of spin.
  Compute the probability of obtaining $-\hbar/2$ for this
  measurement.
  \item Note that you get different answers for parts (a) and (b).
  Mathematically, this comes from the factor of ``$i$'' in the
  equations for $|\mbox{$+y$}\rangle$ and $|\mbox{$-y$}\rangle$.
  Conceptually, this is another example of some quantum weirdness~---
  you would think that a state that is a superposition of stuff solely
  in the z-direction would give the same results for x- and
  y-components, but that isn't the case.  Think about this for a
  moment, and convince yourself that this is weird.
 \end{enumerate}
\label{prob:spin_ii}
\end{problem}


\begin{problem}
Assume that you have an electron that is in a superposition of
``spin-up'' and ``spin-down'' states:
\[ \vert\psi_1\rangle = \sqrt{\frac{4}{5}}\vert\mbox{$-z$}\rangle + i 
\sqrt{\frac{1}{5}}\vert\mbox{$+z$}\rangle.  \]
	\begin{enumerate}
	\item You measure the vertical component of spin. What is the
	probability that you will find $S_z = + \hbar/2$? 
	\item Another electron is in the same state $|\psi_1\rangle$.    You now measure the
	$y$-component of spin. What is the probability that you will find
	$S_y = + \hbar/2$?
	\item Another electron is prepared in a superposition state 
\[ |\psi_2\rangle = \sqrt{\frac{4}{5}} |\mbox{$-z$}\rangle + 
        \sqrt{\frac{1}{5}}\vert\mbox{$+z$}\rangle.  \]	
        What is the probability that a measurement of
	{\em this} electron's $y$-component of spin will find 
        $S_y = + \hbar/2$?
	\item Now, assume that you do find $S_y = + \hbar/2$. You measure $S_y$
	again for the same particle. What is the probability that you'll 
        find $S_y = -\hbar/2$?
        \end{enumerate}
\label{prob:spin_iv}
\end{problem}

\begin{problem}
Assume that a particle is in the state
\[ \vert\mbox{$\psi$}\rangle = \frac{1}{\sqrt{2}} \vert\mbox{$\phi_1$}\rangle + i \frac{1}{\sqrt{2}} \vert\mbox{$\phi_2$}\rangle \]
where $\vert\mbox{$\phi_1$}\rangle$ and $\vert\mbox{$\phi_2$}\rangle$ are a normalized set of basis states.
\begin{enumerate}
\item A measurement is made to determine whether or not the particle is in state $\vert\mbox{$\phi_1$}\rangle$. Calculate the probability of measuring the particle to be in the state $\vert\mbox{$\phi_1$}\rangle$.
\item Let's say that after the measurement made in part (a) the particle {\it is}, in fact, found to be in the state $\vert\mbox{$\phi_1$}\rangle$. The measurement is then immediately repeated.  Calculate the probability of measuring the particle to be in the state $\vert\mbox{$\phi_2$}\rangle$ in this second measurement.
\item Another particle is prepared in the original state $\vert\mbox{$\psi$}\rangle$. A measurement is made on this new particle to determine whether it is in the state $\vert\mbox{$\phi_2$}\rangle$. Calculate the probability of measuring {\it this new} particle to be in the state $\vert\mbox{$\phi_2$}\rangle$.
\end{enumerate}
\end{problem}

\begin{problem}
A particle is in the state given by
\[ \vert\mbox{$\phi$}\rangle = \frac{1}{\sqrt{2}} \vert\mbox{$\psi_1$}\rangle - \frac{i}{\sqrt{2}} \vert\mbox{$\psi_2$}\rangle \]
where $\vert\mbox{$\psi_1$}\rangle$ and $\vert\mbox{$\psi_2$}\rangle$ are a normalized set of basis states.
\begin{enumerate}
\item A measurement is made to determine whether the particle is in state $\vert\mbox{$\psi_1$}\rangle$ or state $\vert\mbox{$\psi_2$}\rangle$. Calculate the probability of measuring the particle to be in the state $\vert\mbox{$\psi_1$}\rangle$.
\item Another particle is prepared in the state $\vert\mbox{$\phi$}\rangle$, and a measurement is made to determine whether the particle is in state $\vert\mbox{$\psi_1$}\rangle$ or state $\vert\mbox{$\psi_2$}\rangle$. Calculate the probability of measuring the particle to be in the state $\vert\mbox{$\psi_2$}\rangle$.
\end{enumerate}
\end{problem}

\begin{problem}
The quantum state of a photon propagating along the z-direction can be written in terms of the states $\vert\mbox{$X$}\rangle$ and $\vert\mbox{$Y$}\rangle$, where $\vert\mbox{$X$}\rangle$ represents a photon polarized along the $x$-direction and $\vert\mbox{$Y$}\rangle$ represents a photon polarized along the $y$-direction. A photon in an arbitray polarization state can be written as a linear superposition of the states $\vert\mbox{$X$}\rangle$ and $\vert\mbox{$Y$}\rangle$ as
\[\vert\theta\rangle = \cos{\theta} \vert\mbox{$X$}\rangle + \sin{\theta} \vert\mbox{$Y$}\rangle \]
where $\theta$ is the angle of polarization of the photon measured with respect to the $x$-axis.
\begin{enumerate}
\item Assuming that the states $\vert\mbox{$X$}\rangle$ and $\vert\mbox{$Y$}\rangle$ form a basis, show that the state $\psi$ is normalized.
\item A polarizer is a device that measures the polarization of a photon along a certain direction.  Let's say that a polarizer is oriented along the $x$-direction, that is, it is measuring for the state $\vert\mbox{$X$}\rangle$. For an incident beam of photons in the state $\vert\mbox{$\theta$}\rangle$ as given above, what is the probability that the polarizer measures the state $\vert\mbox{$X$}\rangle$ (i.e. the photons go through the polarizer)?  What is the state of the photons after making this measurement?
\item Let's say we have a beam of photons in polarization state $\vert\mbox{$X$}\rangle$ incident on a polarizer oriented in a direction making an angle $\theta$ with respect to the $x$-axis.  What is the probability that the photons pass through the polarizer oriented at angle $\theta$? (Hint: think about the previous question in the reverse order of events).  What is the state of the photons after making this measurement?
\item Following up on part (c), the beam in state $\vert\mbox{$\theta$}\rangle$ is incident on a second polarizer oriented along the $y$-axis.  What is the probability that the photons pass through the second polarizer?
\item Now, putting this all together, photons pass through polarizer \#1 oriented along the $x$-axis, then pass through polarizer \#2 oriented at an angle $\theta$, then on to polarizer \#3 oriented along the $y$-axis.  What is the total probability that a photon passing through polarizer \#1 will also pass through polarizer \#3? [Note: Compare your result with what you did in Lab \#18 ``Polarization of Light.''
\end{enumerate}
\end{problem}

\begin{problem}
Using a technique similar to that  used in Example \ref{exam:ztox}, show that you can write the spin-down $S_z$ state $\vert\mbox{$-z$}\rangle$ as a linear superposition of the states $\vert\mbox{$+x$}\rangle$ and $\vert\mbox{$-x$}\rangle$ Compare your results with those given in table (\ref{table:spinTransform}).
\end{problem}

\begin{problem}
Using equations (\ref{eq:plusy}) and (\ref{eq:minusy}), write expressions for the states $\vert\mbox{$+z$}\rangle$ and $\vert\mbox{$-z$}\rangle$ in terms of linear combinations of the states $\vert\mbox{$+y$}\rangle$ and $\vert\mbox{$-y$}\rangle$.  Compare your results with those given in table (\ref{table:spinTransform}).
\end{problem}

\begin{problem}
An electron is placed in a spin state given by
\[ \vert\mbox{$\psi$}\rangle = \frac{1}{2}\vert\mbox{$+z$}\rangle -\frac{\sqrt{3}}{2}\vert\mbox{$-z$}\rangle .\]
\begin{enumerate}
\item Calculate the probability of obtaining a value of $-\hbar/2$ when the $z$-component of spin $S_z$ is measured.
\item Calculate the probability that an electron in state $\vert\mbox{$\psi$}\rangle$ will be measured to have an $x$-component of spin $S_x$ of $+\hbar/2$.
\end{enumerate}
\end{problem}

\begin{problem}
An electron is placed in the spin state
\[ \vert\mbox{$\psi$}\rangle = \sqrt{\frac{2}{3}}\vert\mbox{$+z$}\rangle +\sqrt{\frac{1}{3}}\vert\mbox{$-z$}\rangle \]
as is the case in example \ref{exam:spinsProbabilities}. An experiment is performed to measure the $y$-component $S_y$ of spin.  Calculate the probability that this measurement results in a value of $-\hbar/2$.
\end{problem}

\begin{problem}
Perform the calculation to show that the state $\vert\mbox{$\psi(t)$}\rangle$ given in equation (\ref{eq:plusxTime}) can be rewritten in terms of the states $\vert\mbox{$+x$}\rangle$ and $\vert\mbox{$-x$}\rangle$ as given in equation (\ref{eq:timedependentstate}).
\end{problem}

\begin{problem}
In the last section of this chapter, we found that the time dependent state of a proton in a magnetic field is given by equation (\ref{eq:plusxTime}). Using the same technique as described prior to equation (\ref{eq:yBasisproton}), show that $\vert\mbox{$\psi(t)$}\rangle$ can be written as 
\[ \vert\mbox{$\psi(t)$}\rangle = d_+(t) \vert\mbox{$+y$}\rangle + d_-(t) \vert\mbox{$-y$}\rangle \]
and at the same time show that $d_+(t)$ and $d_-(t)$ are as given in equation (\ref{eq:d-coefficients}).
\end{problem}

\begin{problem}
An electron in a particular system can have any of the discrete
  energies $E_n = (-8.00\mbox{ eV})/n^2$, where $n = 1, 2, 3, \dots$. Assume
  that the electron is in a state 
  \[ |\psi\rangle = 0.7\,|1\rangle + 0.5\,|2\rangle + 0.4\,|3\rangle 
        + 0.3\,|4\rangle + 0.1\,|5\rangle.  \]
  \begin{enumerate}
  \item Show that this state is normalized.
  \item If the energy of the electron is measured, what is the probability that the result of the measurement will be $E_2 = -2.0\units{eV}$? What is the probability of the measured energy being $E_4 = -0.5\units{eV}$?
  \item What is the probability that the result of measuring the energy would be either $E_1$, $E_3$, or $E_5$?
  
  \item Assume that 10,000 electrons are prepared to be in the same
    state $|\psi\rangle$ and that the energy is measured for each of these electrons.  Approximately how many of these electrons will be measured to have energy $E_1$?  $E_2$?  $E_3$?  $E_4$?  $E_5$?
  
  \end{enumerate}
\label{prob:atomic_energies}
\end{problem}

\begin{problem}
For a proton placed in a magnetic field $\vec{B}$ such that it is initially in the state $\vert\mbox{$+x$}\rangle$, the state of the proton at any later time $t$ is given by equations (\ref{eq:timedependentstate}) and (\ref{eq:c-coefficients}).  
\begin{enumerate}
\item Using equations (\ref{eq:timedependentstate}) and (\ref{eq:c-coefficients}), calculate the probability for the proton to be in the state $\vert\mbox{$+x$}\rangle$ at time $t = T/2 = \pi/\omega_o$.
\item Using equations (\ref{eq:timedependentstate}) and (\ref{eq:c-coefficients}), calculate the probability for the proton to be in the state $\vert\mbox{$-x$}\rangle$ at time $t = T/2 = \pi/\omega_o$.
\end{enumerate}
\end{problem}

\begin{problem}
For a proton placed in a magnetic field $\vec{B}$ such that it is initially in the state $\vert\mbox{$+x$}\rangle$, the state of the proton at any later time $t$ is given by equations (\ref{eq:d-timedependenteqn}) and (\ref{eq:d-coefficients}).  
\begin{enumerate}
\item Using equations (\ref{eq:d-timedependenteqn}) and (\ref{eq:d-coefficients}), calculate the probability for the proton to be in the state $\vert\mbox{$+y$}\rangle$ at time $t = 3T/4 = 3\pi/2\omega_o$.
\item Using equations ((\ref{eq:d-timedependenteqn}) and (\ref{eq:d-coefficients}), calculate the probability for the proton to be in the state $\vert\mbox{$-y$}\rangle$ at time $t = 3T/4 = 3\pi/2\omega_o$.
\end{enumerate}
\end{problem}

\begin{problem}
A proton is placed in a magnetic field $\vec{B} = B_o \hat{k}$ oriented alon the $z$-direction with a magnitude $B_o = 13.7 \units{mT}$.
\begin{enumerate}
\item What is the difference in the energy levels of the spin-up $\vert\mbox{$+z$}\rangle$ and spin-down $\vert\mbox{$-z$}\rangle$ states?
\item Determine the angular frequency of precession $\omega_o$ of the proton spin about the magnetic field.
\item Determine the wavelength of the photon needed to cause a transition from the spin-up state to the spin-down state.
\end{enumerate}
\end{problem}

\begin{problem}
Protons placed in a magnetic field can be either in the spin-up $\vert\mbox{$+z$}\rangle$ or spin-down $\vert\mbox{$-z$}\rangle$ state with an energy difference between these two states $\Delta E$.  Incident photons of frequency $2.20 \units{MHz}$ are absorbed causing transitions of the protons from the spin-up $\vert\mbox{$+z$}\rangle$ state to the spin-down $\vert\mbox{$-z$}\rangle$ state.  Determine the magnitude of the magnetic field in which the protons are placed.
\end{problem}
