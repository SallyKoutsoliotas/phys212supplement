\section{Periodic Table}

If it weren't for the Pauli exclusion principle, we wouldn't have the
periodic table and chemistry would be really dull.  Even worse, we
wouldn't be around to notice that chemistry had become dull.

To see why, consider lithium, which has three protons in the nucleus
and three electrons orbiting it.  The first two electrons can go into
the ground state, but Pauli demands that the third electron go into a
new state.  If the lithium atom is in its lowest possible energy
state, that third electron must be in some $n=2$ orbital.  Most of the
chemical behavior of atoms is due to the shape and size of their outer
electron orbitals, so the Pauli exclusion principle is what makes
lithium act very differently than helium.

This process repeats many times over to make up the periodic table.
The more electrons there are, the higher the energy levels must be filled,
similar to the five particle system in Example~\ref{example:five_particles}.
For some specific orbital, meaning some specific values of $n$, $\ell$
and $m_\ell$, there are two possible single-electron states:
$|n\,\ell\,m_\ell\uparrow\rangle$ and
$|n\,\ell\,m_\ell\downarrow\rangle$, and we can put two electrons into
the antisymmetric combination of these two states.  Any additional
electrons will have to go into a new orbital.

