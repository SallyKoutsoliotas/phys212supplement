%%%%%%%%%%%%%%%%%%%%%%%%%%%%%%%%%%%%%%%%%%%%%%%%%%%%%%%%%%%%%%%%%%

\chapter{Quantum Entanglement}
\label{chapter:quantum_entanglement}
%\setcounter{ex}{0}

\section{Introduction}
\label{sec:quantum_entanglement_intro}

In our final chapter on quantum mechanics we introduce the concept of
\textit{entanglement}.  This is a feature of two-particle states (or
multi-particle states) in which the probabilities of the particles are
linked in ways that cannot be described by classical physics.
Entanglement is at the heart of current research in quantum
cryptography, quantum computing, and quantum teleportation.  As we
shall see, it also leads to some extremely counter-intuitive behavior.
We can construct entangled states in which particle 1 is a large
distance away from particle 2, and yet a measurement on particle 1 can
instantly influence particle 2.  Einstein called this ``spooky action
at a distance.''

Finally, we will be able address the question of \textit{completeness}
of quantum states.  When we say a spin has a probability of being
measured in the spin up state, is this probability just a reflection
of our ignorance --- the spin really has some value and we just don't
know it --- or is the particle's spin fundamentally not determined
until we make the measurement?  For decades physicists assumed that
this was an unanswerable question, but in recent years we have found
ways to turn this into an experimental test.  To learn what the
experiments have found, read on!

\section{Entangled Two-Particle States}
\label{sec:two_particle_states}

For a single electron, any possible spin state can be written as
\begin{equation}
\ket{\psi} = c_+\ket{+z} + c_-\ket{-z} = c_+\ket{\uparrow}
 + c_-\ket{\downarrow},
\label{eq:spin_half_state}
\end{equation}
with the appropriate choice of the complex numbers $c_+$ and $c_-$.
For notational simplicity we will switch to writing the spin states in
terms of up and down arrows for the remainder of this chapter.

Now consider a two-particle state made up of two different kinds of
particle.  To be specific, let's take an electron and a positron,
which is a particle with the same mass and spin as the electron, but
it has a $+e$ charge.\footnote{You may have heard of positrons at
  least indirectly: they are used in Positron Emission Tomography,
  also known as a PET scan.}  
What is the expression analogous to Eq.~(\ref{eq:spin_half_state})
for any possible two-particle state?

\begin{table}[t]
\renewcommand{\arraystretch}{1.2}
\begin{center}
\begin{tabular}{lll}
\hline\hline 
state  & electron  & positron \\
\hline
$\ket{\uparrow\uparrow} = 
 \ket{\uparrow}  \ket{\uparrow}$ \quad & 
 $S_z=+\hbar/2$ \quad & $S_z=+\hbar/2$ \\
$\ket{\uparrow\downarrow} =
 \ket{\uparrow} \ket{\downarrow}$ &
  $S_z=+\hbar/2$ & $S_z=-\hbar/2$ \\
$\ket{\downarrow\uparrow} =
 \ket{\downarrow} \ket{\uparrow}$ & 
 $S_z=-\hbar/2$ & $S_z=+\hbar/2$ \\
$\ket{\downarrow\downarrow} =
 \ket{\downarrow} \ket{\downarrow}$ &
  $S_z=-\hbar/2$ & $S_z=-\hbar/2$\\
\hline\hline
\end{tabular}
\caption{The four possible states with definite $S_z$ values.}
\label{table:two_particle_basis}
\end{center}
\end{table}

There are four possible states that have definite values for the
$z$-component of the spin, listed in
Table~\ref{table:two_particle_basis}.  These four states make a
\textit{basis} in the sense that any possible two-particle state for
the electron and positron is a superposition of these states:
\begin{equation}
|\psi\rangle = c_1 \ket{\uparrow\uparrow} +
 c_2 \ket{\uparrow\downarrow} +
 c_3 \ket{\downarrow\uparrow} +
 c_4 \ket{\downarrow\downarrow}  .
\label{eq:general_two_particle_state}
\end{equation}
If the two particles were both electrons, we would need to 
build in indistinguishability by choosing coefficients $c_1$ to $c_4$
that make an anti-symmetric state.  But that's not the case here:
particle 1 is an electron and particle 2 is a positron, so they
are distinguishable from each other.  The state vector does not need
to be anti-symmetric.

We have also introduced in Table~\ref{table:two_particle_basis} the
notation that we can write a two-particle state with definite values
of $S_z$ in a factored form: $\ket{\uparrow\downarrow} =
\ket{\uparrow} \ket{\downarrow}$.  This is
simply two different ways to write the same thing: the electron has
spin up and the positron has spin down.  Both notations have their
advantage and we'll switch back and forth between them.  But be
careful when using the ``factored'' states: the electron state vector
must always be on the left and the positron state vector on the right.
In other words, $\ket{\uparrow} \ket{\downarrow} \neq
\ket{\downarrow} \ket{\uparrow}$.

Now let's examine the behavior of  this two particle state with an
example.  Consider an electron-positron pair in the state
\begin{equation}
\ket{\psi} = 0.9 \ket{\uparrow\uparrow} + 0.1\ket{\uparrow\downarrow}
 + 0.3\ket{\downarrow\uparrow} + 0.3\ket{\downarrow\downarrow}.
\label{eq:example_state}
\end{equation}
First, let's compute the probabilities associated with a measurement of the
positron's $S_z$ spin component.

There are two ways that the positron could be found to have
$S_z=+\hbar/2$, as both the $\ket{\uparrow\uparrow}$ state and the
$\ket{\downarrow\uparrow}$ state have the positron spin up.  So we
need to add together the probability of measuring $\ket{\uparrow\uparrow}$
and the probability of measuring $\ket{\downarrow\uparrow}$:
\begin{align}
\text{Prob(positron spin up)} &= \text{Prob}(\ket{\uparrow\uparrow}) 
+ \text{Prob}(\ket{\downarrow\uparrow}) \nonumber\\
 &=|0.9|^2 + |0.3|^2 \nonumber\\ 
 &= 0.81+0.09 = \boxed{0.90.}
\end{align}
Note that we did not add the coefficients first and then square them.
That would have led to the $|0.9+0.3|^2 = 1.44$, which is a nonsensical
result for a probability!

Similarly, we calculate the probability of the positron being found
with spin down by combining the probabilities of the
$\ket{\uparrow\downarrow}$ and $\ket{\downarrow\downarrow}$ states:
\begin{equation}
\text{Prob(positron spin down)} = |0.3|^2 + |0.1|^2 = \boxed{0.10.}
\end{equation}
Notice that we have a normalized state and the total probability
of the positron having either spin up or spin down
adds to one: $0.90 + 0.10=1$.

Now let's pose a question that gets at the central topic of this
chapter: 

\boxittext{Suppose we start out again with the state $\ket{\psi}$ and
first measure the electron's $S_z$ value.  Will this measurement
affect the state of the positron?}

Here's a way to approach it.  Let's write the basis states in their
factored form, and see if we can combine terms:
\begin{align}
\ket{\psi} &= 
0.9 \ket{\uparrow} \ket{\uparrow}  + 0.1 \ket{\uparrow} \ket{\downarrow} + 
0.3 \ket{\downarrow} \ket{\uparrow} + 0.3 \ket{\downarrow} \ket{\downarrow}
  \nonumber\\
 &= \ket{\uparrow}\Bigl( 0.9\ket{\uparrow} + 0.1\ket{\downarrow} \Bigr)
 + \ket{\downarrow}\Bigl( 0.3\ket{\uparrow} + 0.3\ket{\downarrow} \Bigr) .
\end{align}
Look at what we have done here: we have used a distributive rule to
factor our two-particle state.  This is allowed, but keep in mind that
in any product we must keep the electron state vector on the left and
the positron state vector on the right.

To make this expression more useful, we would
like the terms in parentheses to be normalized states.  We can achieve this
by multiplying and dividing by the appropriate factor, something like
\begin{equation}
\qquad \ket{\psi} = c_+\ket{\uparrow} \biggl(\frac{0.9}{c_+}\ket{\uparrow}
   + \frac{0.1}{c_+}\ket{\downarrow}\biggr)
  + c_-\ket{\downarrow}\biggl( \frac{0.3}{c_-}\ket{\uparrow} + 
  \frac{0.3}{c_-}\ket{\downarrow} \biggr) .
\label{eq:factored_state}
\end{equation}
As you will check in the homework, the terms in parentheses will be
normalized states as long as $c_+ = \sqrt{0.82}$ and
$c_-=\sqrt{0.18}$.  Putting in those values gives
\begin{equation}
\ket{\psi} = \sqrt{0.82} \ket{\uparrow}\ket{\phi_1} +
  \sqrt{0.18}\ket{\downarrow}\ket{\phi_2}
\label{eq:canonical_entangled}
\end{equation}
with the normalized \textit{positron} states
\begin{align}
\ket{\phi_1} &= \frac{9}{\sqrt{82}}\ket{\uparrow} + \frac{1}{\sqrt{82}}
\ket{\downarrow} \nonumber\\
\ket{\phi_2} &= \frac{1}{\sqrt{2}}\ket{\uparrow} + \frac{1}{\sqrt{2}}
\ket{\downarrow} .
\label{eq:positron_states}
\end{align}

Now, $\ket{\psi}$ written in the form
Eq.~(\ref{eq:canonical_entangled}) is nothing other than the
$\ket{\psi}$ we started with in Eq.~(\ref{eq:example_state}).  But it
is now much more useful: we can finally address our central question
(does an electron spin measurement affect the positron state)
using tools we've already learned for states.

An electron $S_z$ measurement will have a probability of $0.82$ of finding
the electron spin up.  If this occurs, the state will collapse to the
new state
\begin{equation}
\ket{\psi_\text{new}} = \ket{\uparrow}\ket{\phi_1}.
\end{equation}
Alternately, there is a probability $0.18$ of the electron being
found to have spin down, in which case the state collapses to
\begin{equation}
\ket{\psi_\text{new}} = \ket{\downarrow}\ket{\phi_2}.
\end{equation}
We can now determine the probabilities for the subsequent positron
spin measurement by looking at the coefficients in the normalized
states $\ket{\phi_1}$ and $\ket{\phi_2}$.  And here is what we find:
\begin{itemize}
\item If we do not measure the electron's spin, the positron has a 
probability $9/10$ of being spin up.
\item If we measure the electron's spin and find it to be spin up, then the
positron has a probability $81/82$ of being spin up.
\item If our electron spin measurement had instead found it to be spin
down, then the positron has a probability $1/2$ of being spin up.
\end{itemize}
This is the essence of entanglement!  Measurements on one particle affect
the state of the other particle.


\section{Separable States}
\label{sec:separable_states}

Not all two particle states are entangled.  An example of a state that
is not entangled would be the following:
\begin{equation}
\ket{\psi} = \frac{1}{2}\ket{\uparrow\uparrow} +
\frac{1}{2}\ket{\uparrow\downarrow} +
\frac{1}{2}\ket{\downarrow\uparrow} +
\frac{1}{2}\ket{\downarrow\downarrow}  .
\label{eq:separable_state}
\end{equation}
Following the procedure of the previous section leads to
\begin{align}
\ket{\psi} &= \ket{\uparrow}
\left(\frac{1}{2}\ket{\uparrow} +\frac{1}{2}\ket{\downarrow}\right)
+ \ket{\downarrow}
\left(\frac{1}{2}\ket{\uparrow} +\frac{1}{2}\ket{\downarrow}\right) 
\nonumber\\
&=\frac{1}{\sqrt{2}}\ket{\uparrow}
\left(\frac{1}{\sqrt{2}}\ket{\uparrow} +\frac{1}{\sqrt{2}}\ket{\downarrow}
 \right)
+\frac{1}{\sqrt{2}}\ket{\downarrow}
\left(\frac{1}{\sqrt{2}}\ket{\uparrow} +\frac{1}{\sqrt{2}}\ket{\downarrow}
 \right)
\end{align}
where in the second line we have made normalized positron states inside
the parentheses.  

And now something interesting has happened: the two positron states
inside parentheses are identical.  This means that if we measure the
electron's spin, the resulting state collapse leads to the same
positron state, regardless of what we obtain for the electron spin.
The positron is no longer entangled with the electron!

Mathematically, we note that having the same positron state in both
parentheses above means we can factor our state even further:
\begin{equation}
\ket{\psi} = 
\left(\frac{1}{\sqrt{2}}\ket{\uparrow} +\frac{1}{\sqrt{2}}\ket{\downarrow}
 \right)
\left(\frac{1}{\sqrt{2}}\ket{\uparrow} +\frac{1}{\sqrt{2}}\ket{\downarrow}
 \right)
\end{equation}
In this form we can see directly that the two-particle state is really
just a product of a single-particle electron state and a single-particle
positron state.  In general, whenever this happens that we can
fully factor the two-particle state into
\begin{equation}
\ket{\psi} = \ket{\phi_\text{electron}}\ket{\phi_\text{positron}} .
\end{equation}
then the particles are not entangled.  We call these
\textit{separable states}.

In summary, two-particle states are separable if they can be fully
factored into a product of single-particle states.  And if they cannot
be factored, then they are entangled.

\section{Bell States}
\label{sec:bell_states}

Not all entangled states are equally entangled.  Some are very nearly
separable: imagine tweaking the coefficients of 1/2 in
Eq.~(\ref{eq:separable_state}) to values like $0.49$ and $0.51$.
Other states are far from separable.

There exist two-particle states that are, in some sense, maximally
entangled.  In these states, which are called Bell
states,\footnote{Named after John Bell, an Irish physicist.  You'll be
  encountering more of his handiwork shortly.}  the $S_z$
value of the positron is completely determined by the $S_z$
measurement on the electron.  One example is the following:
\begin{equation}
  \ket{\psi} = \frac{1}{\sqrt{2}}\ket{\uparrow\uparrow}
 + \frac{1}{\sqrt{2}}\ket{\downarrow\downarrow} .
\label{eq:bell_state_example}
\end{equation}
For an electron-positron pair in this state, an electron spin measurement
is equally likely to give spin up or spin down.  But a subsequent
positron measurement will \textit{always} find the positron to have the same
spin that you just measured for the electron.

By the way, the entanglement goes in both directions: you could
measure the positron spin first and collapse the state of the electron
to be the same spin as the positron.  Also, it is possible to construct
Bell states where the electron and positron spins are exactly opposite.

\section{EPR Paradox}

\label{sec:epr_paradox}

You may have heard at some point that Einstein never accepted quantum
mechanics.  There is a grain of truth in it: Einstein did not dispute
the predictions of quantum theory, but he never believed that quantum
mechanics was a \textit{complete} theory.  For example, he did not
accept the idea that a particle does not have a definite position.  He
was convinced that there must be some more information available that
we simply haven't understood yet, and that if we could work out this
additional aspect to the theory then particles would once again have
precise positions, as in classical physics.  This additional
information, which is not in the quantum theory but Einstein felt must
be a part of reality, is called a \textit{hidden variable}.

In trying to prove his point, Einstein, along with Podolsky and Rosen,
formulated a thought experiment that leads to an apparent
paradox, called the EPR paradox.  
Specifically, consider creating an electron-positron pair in the
Bell state
\begin{equation}
 \ket{\psi} = \frac{1}{\sqrt{2}}\ket{\uparrow\downarrow}
 - \frac{1}{\sqrt{2}}\ket{\downarrow\uparrow}.
\label{eq:epr_bell_state}
\end{equation}
This is fairly easy to do: our particle accelerators create
many electron-positron pairs in their collision events, and conservation
of angular momentum requires that the pair have a total angular momentum
of zero.  Hence, they pop into existence in precisely this Bell state.

Now imagine that you have physically separated the electron-positron
pair without disturbing their spin states.  Maybe, by chance, you get
a pair heading directly away from each other, and you keep them from
interacting with any other particles until they are a full meter away
from each other.

\begin{figure}[h]
\begin{center}
\includegraphics[width=4in]{quantum_entanglement/epr_experiment}
\caption{The EPR thought experiment.  An electron-positron pair is
created in state $\ket{\psi}$ given in Eq.~(\ref{eq:epr_bell_state}).
The particles are physically separated and then their spins are measured by
Stern-Gerlach devices.}
\label{fig:epr_experiment}
\end{center}
\end{figure}

After they are separated, you send the electron through a
Stern-Gerlach device to measure its $S_z$ spin component, as shown in
Fig.~\ref{fig:epr_experiment}.  The result is random, with an equal
chance of spin up or spin down.  But as soon as you have made that
measurement, the positron spin is no longer random.  Essentially, you
have collapsed the state of both particles in that instant, even
though they are separated by a meter.

If that doesn't seem strange, imagine first letting the particles get
a kilometer apart and then doing the measurement.  Or even a light
year apart.  Quantum theory says the distance is irrelevant: if you
can get the particles entangled and keep them entangled, then collapse
of state due to a measurement ``travels'' from one particle to the
other instantly, regardless of their separation.  This is what
Einstein referred to as ``spooky action at a distance.''  And he felt
that this argument proved that quantum mechanics was not a complete
theory.

It is worth emphasizing again that Einstein did not dispute the result
predicted by quantum theory.  But he thought a more natural
explanation was that the state $\ket{\psi}$ in
Eq.~(\ref{eq:epr_bell_state}) is not really a complete description of
the electron-positron pair.  Rather, there is some more information,
the \textit{hidden variable}, which if understood would cause these
measurements to be not mysterious at all.  It helps to have a specific
example of a hidden variable theory in mind to digest this.

\begin{example}{A Hidden Variable Theory}
Imagine that, contrary to what you learned about quantum mechanical
spin in Chapter~\ref{chapter:spin}, an electron really does have a
definite spin vector $\vec S$ with all three components precisely
determined.  None of this probability stuff.  So this $\vec S$ vector
is the hidden variable.  

We can reconcile this idea with the Stern-Gerlach experiment by
claiming that we don't completely understand spin measurement.  When
we measure $S_z$, we somehow end up getting the value $+\hbar/2$ any
time that $\vec S$ has a positive $z$-component.  And we end up
measuring the value $-\hbar/2$ any time $\vec S$ has a negative
$z$-component.
\label{example:hidden_variable_theory}
\end{example}

This simple example of a hidden variable theory explains the EPR
experimental results in a straightforward way.  The electron-positron
pair are created with net angular momentum of zero, so we must have
$\vec S^\text{elec} = -\vec S^\text{pos}$.  These $\vec S$ vectors are
established at the beginning, when the two particles are in contact.
And while we don't know the value of
either $\vec S^\text{elec}$ or $\vec S^\text{pos}$,
we do know their $z$-components have
opposite sign.  So the EPR experiment will always give opposite signs
for the electron and positron $S_z$ components, but not due to any
communication at a distance.  

This is a pretty compelling argument!  But interestingly, the EPR
paradox did not sway most physicists to abandon quantum mechanics.
One reason was that Bohr showed that this instantaneous state collapse
across a large distance did not actually violate special relativity.
No actual object travels from the electron to the positron.  But more
significantly, no actual information travels either: the person
measuring the electron cannot send a signal via the state collapse
(``spin up if by land, spin down if by sea'') because they cannot
control what they will find for the electron spin.  They just get a
sequence of random results, and the positron measurer also gets a
sequence of random results.  Comparison later reveals that the random
results are connected:  every time the electron observer measures $S_z^\text{elec}$ to be spin up, the positron observer measures $S_z^\text{pos}$ to be spin down and visa versa.

The main reason that the EPR paradox did not push physicists to
abandon quantum mechanics, though, is that it didn't seem to matter.
After all, the hidden variable theories predicted the same results as
quantum mechanics.  Since physicists knew how to use quantum mechanics
and it kept giving successful predictions about the microscopic world,
there was little motivation to worry about whether we should replace
quantum mechanics with a hidden variable theory.

\section{Bell's Proposed Experiment}
\label{sec:bell_experiment}
\label{section:bell}

Bell changed all that.  To everyone's great surprise, in 1964 he
constructed a thought experiment where quantum mechanics and hidden
variable theories actually predict different results.  This thought
experiment suddenly made it an experimental question, so nature could
tell us who was right!

Bell's clever idea rested on a variation of the EPR thought experiment
where the Stern-Gerlach device that is measuring the positron spin is
rotated.  The electron's spin is still measured the same way, so we
will find $S_z = +\hbar/2$ or $-\hbar/2$ as before.  But the
positron's spin is measured along some direction $\hat n$ that makes
an angle $\theta$ with respect to the $z$ axis, as shown in
Fig.~\ref{fig:bell_experiment}.  For simplicity, we will take
$\theta = 45^\circ$.

\begin{figure}[t]
\begin{center}
\includegraphics[width=4.5in]{quantum_entanglement/bell_experiment}
\caption{The Bell thought experiment, which is the same as the EPR
experiment except that the positron Stern-Gerlach detector is rotated.}
\label{fig:bell_experiment}
\end{center}
\end{figure}

\subsection{QM Prediction for the Bell Experiment}

If we rotate our positron detector to be $45^\circ$ from the $z$-axis
and measure this component of the spin, call it $S_{45^\circ}$, what
possible values will we find?  Just like with any spin component,
we'll find $+\hbar/2$ and $-\hbar/2$.  And after the measurement, the
spin will be in either the $\ket{\nearrow}$ state or the
$\ket{\swarrow}$ state, just as an $S_x$ measurement results in either
the $\ket{+x}$ or $\ket{-x}$ states.

We can relate our familiar up and down spins to this new pair of
states:
\begin{align}
 \ket{\uparrow} &=  b_+\ket{\nearrow} + b_-\ket{\swarrow}
  &
 \ket{\downarrow} &=  b_-\ket{\nearrow} - b_+\ket{\swarrow}.
  \label{eq:rotated_basis}
\end{align}
The coefficients, $b_+$ and $b_-$ are given by
\begin{align}
 b_+ &= \sqrt{\frac{1+\cos \theta}{2}}  &
 b_- &= \sqrt{\frac{1-\cos \theta}{2}},  
  \label{eq:b_variables}
\end{align}
where $\theta$ is measured from the $z$-axis as shown in 
Fig.~\ref{fig:bell_experiment}. 

We'll create an electron-positron pair in the initial state given by
Eq.~(\ref{eq:epr_bell_state}).  Let's explore what quantum mechanics
predicts for the Bell Experiment.  We're going to consider the
case where an observer measures $S_z^\text{elec}$ to be spin down.
According to quantum mechanics, a measurement of $S_z^\text{elec}$
= $-\hbar/2$ collapses the state of our electron-positron pair to
be $\ket{\psi} = \ket{\downarrow\uparrow}$.  To put it another way,
if we measure the state of the electron to be $\ket{\downarrow}$,
quantum entanglement says that the state of the positron must be
$\ket{\uparrow}$.  The subsequent positron measurement will find
$S_{45^\circ}$ to be either $+\hbar/2$ or $-\hbar/2$ with probabilities
that depend on the $b_+$ and $b_-$ variables.  Since the positron state is
$\ket{\uparrow}$, the probability of measuring $S_{45^\circ}^\text{pos}$
= $+\hbar/2$ is $\left|b_+\right|^2$ and the probability of measuring
$S_{45^\circ}^\text{pos}$  = $-\hbar/2$ is $\left|b_-\right|^2$.  In your
homework, you will calculate these probabilities.

\subsection{Hidden Variable Prediction for the Bell Experiment}

\label{subsec:hidden_variable_for_bell}

Now here is where it gets interesting: Bell constructed a proof that
\textit{any} hidden variable theory\footnote{Technically, any
\textit{local} hidden variable theory, to satisfy the lawyers and
mathematicians.} with the detectors oriented as in
Fig.~\ref{fig:bell_experiment} and with $\theta=45^\circ$ will result
in probabilities that are incompatible with the predictions of quantum
mechanics.  So, either the hidden variable theory or quantum mechanics
must be wrong!

While we won't provide a proof of Bell's theorem in full generality
here, we will show how to determine the predictions of
hidden variable theory introduced in
Example~\ref{example:hidden_variable_theory} of
Section~\ref{sec:epr_paradox}.  According to this hidden variable theory,
the electron and positron have fully determined spin values $\vec S^\text{elec}
 = -\vec S^\text{pos}$, but we just don't know them. 

\begin{figure}[h]
\begin{center}
\includegraphics[width=1.4in]{quantum_entanglement/hidden_variable_calculation}
\caption{The circle represents all possible $\vec S^\text{pos}$
  directions.  After a measurement of $S_z^\text{elec}=-\hbar/2$, the
  possible directions of $\vec S^\text{pos}$ are limited to the upper
  hemisphere (shaded region).}
\label{fig:hidden_variable_calculation}
\end{center}
\end{figure}

Picture the positron spin variable $\vec S^\text{pos}$.  Since we
don't know its value and we aren't in any way controlling how it is
set when the electron-positron pair are created, it is reasonable to
assume it is equally likely to be pointing in any direction.  We can
represent all possible directions of $\vec S^\text{pos}$ by the
surface of a sphere, where the vector $\vec S$ has its tail on the
origin.  A circular cross section of this sphere is shown in
Fig.~\ref{fig:hidden_variable_calculation}.  The dashed line in the
figure is the boundary between the positive and negative values for
$S_{45^\circ}^\text{pos}$.  That is, for spins $\vec
S^\text{pos}$ above and to the right of the dashed line a measurement
would result in $S_{45^\circ}^\text{pos}$  = $+\hbar/2$.  The region below
and to the left corresponds to $S_{45^\circ}^\text{pos}$  = $-\hbar/2$.

Now consider the case where we have measured the electron's $z$-component
of spin and found $S_z^\text{elec}$ = $-\hbar/2$.  This happens half
the time, and when it happens the positron $\vec S^\text{pos}$ must
be somewhere in the upper hemisphere, shown as the shaded region in
Fig.~\ref{fig:hidden_variable_calculation}.  This agrees with the QM
prediction so far, where if we measure $S_z^\text{elec}$ = $-\hbar/2$, we
know that we would also measure $S_z^\text{pos}$ = $+\hbar/2$.  But, here,
rather than the positron actually \emph{being in a spin up state}, the
hidden variable theory says that the original $\vec S^\text{pos}$ should
determine the probabilities for measuring  $S_{45^\circ}^\text{pos}$ as
spin up or down.  According to the figure, a fraction 3/4 of the shaded
region results in a value $S_{45^\circ}^\text{pos}$  = $+\hbar/2$.
Bell constructed a proof that for \emph{any} hidden variable theory,
this fraction of 3/4 is the \emph{maximum} probability of measuring
$S_{45^\circ}^\text{pos}$ = $+\hbar/2$.

So, let's recap.  We create an electron-positron pair in the state:
\begin{equation}
 \ket{\psi} = \frac{1}{\sqrt{2}}\ket{\uparrow\downarrow}
 - \frac{1}{\sqrt{2}}\ket{\downarrow\uparrow}.
\end{equation}
We make a measurement of the $z$-component of the electron's spin,
and we find that the electron is spin down.  Quantum mechanics says
that the positron is thus in the state $\ket{\uparrow}$, which means
the probability of measuring $S_{45^\circ}^\text{pos}$  = $+\hbar/2$
is $\left|b_+\right|^2$, or 85\%.  The hidden variable theory says that
the positron spin is determined by its original $\vec S^\text{pos}$, and
we have a $\le$75\% probability of measuring $S_{45^\circ}^\text{pos}$
= $+\hbar/2$.

\bigskip

\noindent\dots so which is right?

\section{The Aspect Experiment}
\label{sec:aspect_experiment}

These experiments are quite hard to do.  In fact, the best
experimental tests of Bell's inequality to date is done not with
electrons and positrons, but rather with polarization states of
photons.  The first definitive test was performed in the 1980s in
France by the research group of Aspect, and their data came
down firmly on the side of \dots drum roll \dots quantum mechanics!  

Perhaps you have been holding out hope that the spin of an electron
could have some precise value and that quantum mechanics was just an
incomplete theory, but Bell and Aspect have ruled out that possibility!
It appears that quantum mechanics is a complete theory, and quantum
mechanics states really do represent the most knowledge possible about
the condition of a particle.  The randomness of uncertainty in quantum 
mechanics that we've been studying in this unit \emph{is real.}

%\footnote{There are a few physicists still concerned that
%the experiments haven't closed all the loopholes.  But most physicists at
%this point are convinced by the work of Aspect and others.}

\newpage

\section*{Problems}
\label{sec:quantum_entanglement_problems}
\markboth{PROBLEMS}{PROBLEMS}

\begin{problem}
  \label{prob:entangled_i}
  An electron-positron pair is in the state
  \[
  \ket{\psi} = \frac{1}{\sqrt{10}}\ket{\uparrow\uparrow}
  +\frac{1}{\sqrt{5}}\ket{\uparrow\downarrow} +
  \frac{1}{\sqrt{5}}\ket{\downarrow\uparrow} +
  \frac{1}{\sqrt{2}}\ket{\downarrow\downarrow} .
  \]
  \begin{enumerate}
  \item Calculate the probability that a measurement of the electron
    $S_z$ spin component will give the value $+\hbar/2$.
  \item For the same state $\ket{\psi}$, calculate the probability
    that a measurement of the positron $S_z$ component will give the
    value $-\hbar/2$.
  \end{enumerate}
\end{problem}



\begin{problem}
  Show that the state $\ket{\psi}$ in
  Eq.~(\ref{eq:factored_state}) takes the form shown in
  Eq.~(\ref{eq:canonical_entangled}), with positron states
  $\ket{\phi_1}$ and $\ket{\phi_2}$ given by
  Eq.~(\ref{eq:positron_states}).
\end{problem}

\begin{problem}
  \label{prob:entangled_ii}
  An electron-positron pair is in the state
  \[
  \ket{\psi} = \frac{1}{\sqrt{10}}\ket{\uparrow\uparrow}
  +\frac{1}{\sqrt{5}}\ket{\uparrow\downarrow} +
  \frac{1}{\sqrt{5}}\ket{\downarrow\uparrow} +
  \frac{1}{\sqrt{2}}\ket{\downarrow\downarrow} .
  \]
  \begin{enumerate}
  \item Convert this state to the form $\ket{\psi} =
    c_+\ket{\uparrow}\ket{\phi_1} + c_-\ket{\downarrow}\ket{\phi_2}$.
    Determine the coefficients $c_+$ and $c_-$ and the normalized
    positron states $\ket{\phi_1}$ and $\ket{\phi_2}$.
  \item Check that your coefficients satisfy $|c_+|^2+|c_-|^2=1$.
  \item Is your answer consistent with Problem
    \ref{prob:entangled_i}(a)?
  \end{enumerate}
\end{problem}



\begin{problem}
An electron-positron pair is in the state
\[
\ket{\psi} = 0.6 \ket{\uparrow} \ket{\phi_1} 
  + 0.8 \ket{\downarrow} \ket{\phi_2}
\]
with positron states
\[
\ket{\phi_1} = 
\sqrt{\frac{1}{3}} \ket{\uparrow} - \sqrt{\frac{2}{3}} \ket{\downarrow} 
\qquad\quad
\ket{\phi_2} = 
\sqrt{\frac{2}{3}} \ket{\uparrow} + \sqrt{\frac{1}{3}} \ket{\downarrow}. 
\]

\begin{enumerate}
\item What is the probability that a measurement of the $z$-component
  of the \textbf{positron's} spin will find a result $S_z^\text{pos} =
  +\hbar/2$?

\item You now measure the $z$-component of the \textbf{electron's}
  spin and find a value $S_z^\text{elec} = +\hbar/2$. Write down the new
  state $\ket{\psi_\text{new}}$ of the electron-positron system
  immediately after this measurement.

\item What is the probability that a measurement of the $z$-component
  of the positron's spin will now find a result of $S_z^\text{pos} =
  +\hbar/2$? Is it the same as your answer to part (a)?

\item Is the original state $\ket{\psi}$ an entangled state? How can
  you tell?
\end{enumerate}
\end{problem}

 
\begin{problem}
  Construct a separable state that has all four coefficients in
  Eq.~(\ref{eq:general_two_particle_state}) not equal to zero.
\end{problem}


\begin{problem}
  Two Bell states were provided in the reading, in
  Eqs.~(\ref{eq:bell_state_example}) and (\ref{eq:epr_bell_state}).
  Find two more Bell states.
\end{problem}



\begin{problem}
  Let's verify the details for the quantum mechanics prediction
  for the Bell experiment with the positron detector rotated
  $45^\circ$ from the $+z$ direction.  We'll use a simulation of the 
  Stern-Gerlach Experiment.
 \begin{enumerate}
  \item Go to \verb+https://phet.colorado.edu/sims/stern-gerlach/stern-gerlach_en.html+
  \item In the simulation rotate the angle to $45^\circ$, and change
  the spin orientation to ``$+z$''.  You now have the simulation set up
  to take atoms with $S_z$ = $+\hbar/2$ and measure the spin of these
  particles along a $45^\circ$ angle.
  \item To get a feel for how this works, try firing a few atoms.
  You'll notice a counter at the bottom of the page keeping track of how
  many atoms have been measured with spin up or spin down as measured
  along a $45^\circ$ angle.
  \item  Turn on ``AutoFire'' and keep it running until the percentages
  are no longer changing. (Increase the speed of the auto fire so that you 
don't have to wait forever.)  Based on the results of the simulation,
  what is the probability of measuring $S_{45^\circ}$  = $+\hbar/2$?
  \item Now, using Eq.~(\ref{eq:rotated_basis}), calculate the probability
  of measuring a particle to have $\ket{\nearrow}$ if it started in
  the state $\ket{\uparrow}$.  Do your results agree with those of
  the simulation?
 \end{enumerate}
  %In the case that the electron
  %has been measured to be spin down, the probabilities for the
  %positron measurement are given in
  %Eq.~(\ref{eq:bell_qm_probabilities}).
%  \begin{enumerate}
%  \item Go through the steps to show that the Bell state,
%    Eq.~(\ref{eq:epr_bell_state}), with the substitutions of
%    Eq.~(\ref{eq:rotated_basis}) results in
%    Eq.~(\ref{eq:bell_exp_state}).
%  \item Use Eq.~(\ref{eq:bell_exp_state}) and
%    Eq.~(\ref{eq:spin_product_average}) to calculate $\langle \hat
%    S_z^\text{elec} \hat S_{45^\circ}^\text{pos}\rangle$.  Check that
%    your answer matches Eq.~(\ref{eq:bell_qm_average}).
%  \end{enumerate}
\label{prob:epr_sim}
\end{problem}

\begin{problem}
  Einstein died in 1955, before Bell had derived his theorem and well
  before Aspect did the experiments that ruled out hidden variable
  theories.  It is often stated that refusing to accept quantum
  mechanics is one of Einstein's mistakes.  What do you think of that
  statement?
\end{problem}


\newpage 

%\begin{problem}
%  Go through the details to show that $\langle \hat S_z^\text{elec}
%  \hat S_{45^\circ}^\text{pos}\rangle = -1/2$ for the hidden variable
%  theory presented in Section~\ref{subsec:hidden_variable_for_bell}.
%\end{problem}
\begin{problem}
Suppose we modified Bell's experiment and aligned the positron detector to be at a $60^\circ$ angle with respect to the $z$-axis.  
\begin{enumerate}
  \item In quantum mechanics, if the electron is measured to be spin
    down ($S_z=-\hbar/2$), what is the probability that the
    positron will be found to have $S_{60^\circ} =+\hbar/2$?  There are two ways to approach this.  You can use Eqs.~(\ref{eq:rotated_basis}) 
and (\ref{eq:b_variables}).  You can also use the same simulation as in Problem 
\#\ref{prob:epr_sim}.
%  \item Based on your answer to part (a), what would be expect for the
%    average $\langle S_z^\text{elec} S_x^\text{pos}\rangle$ according to quantum
%    mechanics.
  \item Now we turn to the hidden variable theory of
    Section~\ref{sec:bell_experiment}.  In this case, if the electron is
    measured to be spin down what is the maximum probability that the
    positron will be found to have $S_{60^\circ}=+\hbar/2$?
  \item Note the difference between the predictions of quantum mechanics 
    versus the hidden variable theory in this case.
\end{enumerate}
\end{problem}



\begin{problem}
  Explain why hidden variable theories do not require ``spooky action 
at a distance.''
\end{problem}


%%%%%%%%%%%%%%%%%%%%%%%%%%%%%%%%%%%%%%%%%%%%%%%%%%%%%%%%%%%%%%%%%%%



\begin{problem}  
  Show that the state
  \[
  \ket{\psi} = \frac{1}{2}\ket{\uparrow\uparrow}
  -\frac{1}{2}\ket{\uparrow\downarrow}
  +\frac{1}{2}\ket{\downarrow\uparrow}
  -\frac{1}{2}\ket{\downarrow\downarrow}
  \]
  is separable and can be written as $\ket{\psi} =
  \ket{\phi_\text{electron}} \ket{\phi_\text{positron}}$.  Find
  $\ket{\phi_\text{electron}}$ and $\ket{\phi_\text{positron}}$.
\end{problem}


\begin{problem}  
  An electron-positron pair is in the state
  \[
  \ket{\psi} = 0.5\ket{\uparrow\uparrow} + 0.1\ket{\uparrow\downarrow}
  + 0.7\ket{\downarrow\uparrow} + 0.5\ket{\downarrow\downarrow} .
  \]
  \begin{enumerate}
  \item Convert this state to the form $\ket{\psi} =
    c_+\ket{\uparrow}\ket{\phi_1} + c_-\ket{\downarrow}\ket{\phi_2}$.
    Determine the coefficients $c_+$ and $c_-$ and the normalized
    positron states $\ket{\phi_1}$ and $\ket{\phi_2}$.
  \item Based on your answer to part (a), is $\ket{\psi}$ entangled or
    not?  Describe how you can tell.
  \end{enumerate}
\end{problem}

\newpage 

\begin{problem}
  An electron-positron pair is in the state
  \[
  \ket{\psi} = \frac{1}{\sqrt{10}}\ket{\uparrow\uparrow}
  +\frac{1}{\sqrt{5}}\ket{\uparrow\downarrow} +
  \frac{1}{\sqrt{5}}\ket{\downarrow\uparrow} +
  \frac{1}{\sqrt{2}}\ket{\downarrow\downarrow}
  \]
  which is the same state as in Problem~\ref{prob:entangled_ii}.
  \begin{enumerate}
  \item Determine the probability of a positron $S_z$ measurement
    finding the value $+\hbar/2$.
  \item Starting from state $\ket{\psi}$, an electron $S_z$
    measurement has found the value $+\hbar/2$.  Using your results
    from Problem~\ref{prob:entangled_ii}, calculate the probability of
    the positron being spin up.
  \item Starting from state $\ket{\psi}$, an electron $S_z$
    measurement has found the value $-\hbar/2$.  Using your results
    from Problem~\ref{prob:entangled_ii}, calculate the probability of
    the positron being spin up.
  \end{enumerate}
\end{problem}


\begin{problem}
  For each of the following states, determine whether they are
  separable or entangled.
  \begin{enumerate}
  \item $\ket{\uparrow\uparrow}$
  \item $\ket{\downarrow\downarrow}$
  \item $\displaystyle\frac{1}{\sqrt{2}}\ket{\uparrow\uparrow} +
    \frac{1}{\sqrt{2}} \ket{\downarrow\downarrow}$
  \end{enumerate}
\end{problem}


\begin{problem}
  Suppose we modified Bell's experiment and aligned the positron
  detector to be in the $+x$ direction.  That is, we will measure the
  $S_x$ value for the positron.
  \begin{enumerate}
  \item In quantum mechanics, if the electron is measured to be spin
    down ($S_z=-\hbar/2$), what are the probabilities that the
    positron will be found to have $S_x=\pm\hbar/2$?
%  \item Based on your answer to part (a), what would be expect for the
%    average $\langle S_z^\text{elec} S_x^\text{pos}\rangle$ according to quantum
%    mechanics.
  \item Now we turn to the hidden variable theory of
    Section~\ref{sec:bell_experiment}.  In this case, if the electron is
    measured to be spin down what are the probabilities that the
    positron will be found to have $S_x=\pm\hbar/2$?
  \item Will this modification of Bell's experiment allow us to
    distinguish between quantum mechanics and our hidden variable
    theory?
  \end{enumerate}
\end{problem}


\vfill 

\thispagestyle{empty}
%%%%%%%%%%%%%%%%%%%%%%%%%%%%%%%%%%%%%%%%%%%%%%%%%%%%%%%%%%%%%%%%%%%%%%%%

