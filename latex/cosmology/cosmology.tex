
%%%%%%%%%%%%%%%%%%%%%%%%%%%%%%%%%%%%%%%%%%%%%%%%%%%%%%%%%%%%%%%%%%%

\chapter{Cosmology}
\label{chapter:cosmology}
%\setcounter{ex}{0}


\section{Introduction}
\label{sec:cosmology_intro}

Cosmology is the study of the origins of the cosmos, that is, the
history of the universe.  The present standard model of cosmology is
the subject of this chapter.

At the end of Chapter~\ref{chapter:interactions}, we claimed that the
temperature and available energy in the early universe were sufficient
to achieve universal symmetry.  In this chapter we present a simple
relation that shows how energy and temperature decrease with time.  We
then look in detail at each of the major events that served to mark
transitions from one phase of the universe to another.  We'll learn
why these events occurred when they did, and what ramifications these
events had for the interactions that followed.  And finally we'll see
what observational evidence exists that could test the microscopic
theories of particle physics.

\section{The Early Universe}
\label{sec:early_universe}

The standard theory of cosmology is the Big Bang model, in which the
universe began as a hot dense ball of stuff and has been expanding and
cooling ever since.  The first and primary evidence for this model was
Edwin Hubble's discovery in the 1920's that the universe is expanding.
From analysis of Doppler shifted spectra and other observations,
Hubble showed that most distant galaxies are receding from our
galaxy. Furthermore, their velocity of recession is roughly
proportional to their distance from us.

To see how this could happen, imagine baking a giant loaf of raisin
bread.  As the dough expands, each raisin gets farther from all the
others and, if the expansion is uniform, the recessional rates at any
instant of time are proportional to the separations.  The recession
rates of galaxies in the expanding universe are similarly proportional
to the separation between galaxies. The proportionality can be
expressed simply as
\begin{equation}
V = HR  \qquad\mbox{(Hubble's Relation)}
\end{equation}
where $V$ is the velocity of recession, $R$ is the galaxy distance and
$H$ is Hubble's constant.  By constant, we mean it is the same number
for many different galaxies, not that it doesn't change with time.  In
fact we argue below that it must be decreasing with time.  Its current 
accepted value is $20.8\units{km/s}$ per million light-years.  
This means that if
% Planck Mission value of H is 67.8 \pm 0.77 km/s/Mpc, or 
% 20.8 \pm 0.24 km/s/Mly
%    ML April 2016
%IF YOU UPDATE THIS, PLEASE UPDATE THE VALUE 2 LINES DOWN TOO.  
% H is now 70 km/s/Mpc, or 21 km/s/Mly -NFL Dec07
%value is about $15\, \mbox{km/s}$ per million light-years.  This means that if
galaxy $B$ is a million light years more distant than galaxy $A$, then $B$
is receding $20.4\units{km/s}$ faster than $A$.

Okay, now here is the key point: if the universe is expanding, then it
is also cooling down.  Especially when considering the early moments
of the universe, you can compare it to a gas that is undergoing an
adiabatic expansion (no heat flow into or out of the system, which is
the entire universe in this case).  Recall from PHYS 211 that a gas
cools down when undergoing an adiabatic expansion.  It can be shown
that\footnote{``It can be shown that'' is physics-lingo for ``the
derivation is a royal pain-in-the-neck, so we will just skip it
here.''} the temperature $T$ of the universe during its early stages
--- during the ``radiation-dominated'' epochs, i.e., the first million
years --- is given roughly by the equation
\begin{equation}
T = \frac{10^{10}}{\sqrt t},
\label{eq:early_univ_temp}
\end{equation}
where time $t$ is in seconds and temperature $T$ is in Kelvin.  Even
though the assumptions of this derivation\footnote{Which we haven't
told you.}  are very much oversimplified,
Eq.~(\ref{eq:early_univ_temp}) gives results correct within a factor
of 2 or 3 up to $t = 10^6$ years.  After that time, the universe is
cool enough to start acting like a collection of material particles
rather than a bunch of photons.\footnote{One of the assumptions that we
haven't explained.}  Thus, for times near the present,
Eq.~(\ref{eq:early_univ_temp}) overestimates the temperature of
radiation considerably.

A useful relation can be obtained from Eq.~(\ref{eq:early_univ_temp}).
Recall from kinetic theory (PHYS 211) that for particles in
equilibrium, the average kinetic energy equals $\frac{3}{2}kT$, where
Boltzmann's constant $k= 8.63\times 10^{-5}\units{eV/K}$.  If we round these
figures to the same accuracy as Eq.~(\ref{eq:early_univ_temp}), we
find $E \simeq kT$ with $k \simeq 10^{-4}\units{eV/K}$, and thus typical
particle energies of the early universe are given by
\begin{equation}
\label{eq:early_univ_energy}
E \simeq kT \simeq \frac{10^6}{\sqrt t},
\end{equation}
where energy $E$ is in eV and time $t$ is again in seconds.
Equations~(\ref{eq:early_univ_temp}) and (\ref{eq:early_univ_energy})
are handy relations that will be used throughout the chapter.

\section{Crucial Events of the Early Universe}
\label{sec:crucial_events}

The universe began some 14 billion years ago with a Big Bang and
has been expanding and cooling ever
since. Table~\ref{table:early_universe} is a list of the crucial
events that occurred early on, most of which we will try to explain.

How can we possibly determine when these events occurred? Except for
the end of neutrino interactions and the time of quark confinement,
each of these events can be understood in terms of the energies
available to maintain thermal equilibrium.  For instance, the end of
the electro-weak unification occurred when the massive gauge bosons of
electro-weak theory (the $W^+$, $W^-$, and $Z^0$) could no longer be
produced in prodigious amounts from random collisions of particles in
thermal equilibrium.  In other words, as the universe cooled, the
ambient thermal energy became insufficient to create such massive
particles.  When this occurred, the large mass of the $W$ and $Z$
differentiated them from the massless photon, the messenger particle
of electromagnetic theory.  After this time the weak and
electromagnetic interactions were observed to behave differently.

\begin{table}[tbp]
\caption{Events of the early universe.}
\label{table:early_universe}
\begin{center}
\begin{tabular}[tbp]{cl}
Time & Event \\
\hline\hline
$0$          & Big Bang \\
$10^{-43}\units{s}$ & Planck Time --- end of Universal Symmetry\\
$10^{-34}\units{s}$ & End of Grand Unification\\
$10^{-10}\units{s}$ & End of Electro-weak Unification\\
$10^{-4}\units{s}$  & Confinement of quarks into hadrons\\
$1\units{s}$        & Neutrinos stop interacting (decouple)\\
$400\units{s}$      & Protons and neutrons combine to form nuclei\\
$10^5\units{yr}$    & Nuclei and electrons combine to form atoms\\
\hline
\end{tabular}
\end{center}
\end{table}


So how can we relate the time of the splitting off of the weak
interaction to properties of the $W$'s and $Z$'s?  Here's how. The
theoretical model for electro-weak unification, the Weinberg-Salam
theory of 1973, predicted rest energies for $W$ and $Z$ in the range
of 80--$100\units{GeV}$.  This prediction was dramatically confirmed when the
rest energies of $W$ ($84\units{GeV}$) and $Z$ ($92\units{GeV}$) were 
measured in the
process of their experimental discovery at CERN in 1983.  If we round
these rest energies up to $E=100\units{GeV}=10^{11}\units{eV}$, we can use
Eq.~(\ref{eq:early_univ_energy}) to estimate the time when this much
energy was available as ambient thermal energy.  We find
\begin{equation}
t = \left(\frac{10^6}{E}\right)^2 = \left(\frac{10^6}{10^{11}}\right)^2 
   = 10^{-10}\units{s}
\end{equation}
as shown in the table. In general, the energy available at the time of
all but two of the events above is related either to the rest energy
of a particle or to a reaction energy associated with the event.  The
problems expand on this relationship.

Finally, let's try to explain the remaining two events: the decoupling
of neutrinos from thermal equilibrium with the rest of the universe
and the confinement of quarks inside hadrons.  Both of these events
depend on the average density of the universe which, as the universe
expands, is continually decreasing.

Using some of the assumptions needed (but not presented here) to
derive Eqs.~(\ref{eq:early_univ_temp}) and
(\ref{eq:early_univ_energy}), along with estimates of the present
average density of the universe, we find a density of $4\times
10^5\units{g/cm$^3$}$ (and thus $4\times 10^5$ times that of water) for
neutrino decoupling and $4\times 10^{13}\units{g/cm$^3$}$ for quark
confinement.  Now neutrinos are very penetrating particles. Those
produced in modern accelerators can penetrate miles of plate steel
with less than one in a million being absorbed!  The high energy
neutrinos of the early universe were even more penetrating.  When the
universe's density fell to $4\times 10^5\units{g/cm$^3$}$, the absorption
rate of neutrinos (in reactions like $\nu+p \to n + e^+$) fell
significantly below their production rate, and neutrinos could no
longer exchange enough energy with the rest of the universe to
maintain thermal equilibrium.

Quark confinement is explained by density considerations in a similar
way.  Before $10^{-4}\units{s}$, individual quarks were ``free'' in
the sense that they did not travel around through space in groups of
quark-quark-quark (baryons) or quark-antiquark (mesons).  A single
colored quark could go where it pleased.  But this was only because
the density of the universe was comparable to the density of an atomic
nucleus, around $10^{14}\units{g/cm$^3$}$.  The entire universe was
one big nucleus!  As a result, any individual quark was never far from
lots of other quarks, and the gluon ``rubber band'' never had to be
stretched to give rise to confining forces. After this time, the
density decreased enough that spaces started forming between nucleons
and other particles, and quarks became confined inside hadrons in
colorless combinations.

\section{Epochs of the Early Universe}
\label{sec:early_universe_epochs}

Let's now see what was going on in the universe during each epoch, the
intervals between the crucial events of
Section~\ref{sec:crucial_events}, and explain briefly why.

\begin{description}

\item[Bang to $\mathbf{10^{-43}}\units{s}$:] Very speculative.  This would be
  the time of universal symmetry, with quantized gravity incorporated
  with the other three interactions.  Nobody really knows how.

\item[$\mathbf{10^{-43}}$ to $\mathbf{10^{-34}}\units{s}$:] Grand
  unification prevails.  The strong, weak and electromagnetic
  interactions are on equal footings.  Quarks and leptons are not yet
  distinct, since there are plenty of $X$ bosons to facilitate
  interconversion.

\item[$\mathbf{10^{-34}}$ to $\mathbf{10^{-10}}\units{s}$:] The strong
  interaction acts separately from the electro-weak.  The $X$ bosons
  are now too heavy to be produced in random collisions and they decay
  away quickly leaving distinct quarks, antiquarks, leptons and
  antileptons.  However, due to a very subtle flaw in symmetry (which
  we won't explain here), the $X$ boson and its antiparticle
  ($\overline X$) decay slightly differently.  This leaves a slight
  excess (about one part in $10^9$) of quarks over antiquarks.  The
  universe now consists of a hot ``soup'' of quarks, leptons, and
  gauge bosons ($W$'s and $Z$'s, gluons, photons) all in exact thermal
  equilibrium.

\item[$\mathbf{10^{-10}}$ to $\mathbf{10^{-4}}\units{s}$:] The weak
  interaction now acts separately from electromagnetism, because $W$'s
  and $Z$'s are no longer produced abundantly.  By the end of this
  epoch, the density of the universe has fallen sufficiently that
  quarks begin collecting into hadrons.

\item[$\mathbf{10^{-4}}$ to 1 s:] With quarks permanently confined,
  the ``soup'' now consists of hadrons and leptons.  Energies are now
  insufficient to create exotic particles or to produce hadronic
  particle-antiparticle pairs. Thus annihilation of antiparticles
  occurs rapidly and essentially all baryonic antimatter is wiped out,
  leaving the small original (since the time $t=10^{-34}\units{s}$) excess of
  matter.  We are left with primarily protons, neutrons, electrons,
  positrons, neutrinos and photons in thermal equilibrium.  So much
  radiation is released from the annihilation of quarks that there are
  approximately $10^9$ photons/baryon, the value we observe today.

\item[1 to 400 s:] The universe is now diffuse enough (``only'' about
  400,000 times as dense as water) that neutrinos don't interact
  enough to stay in equilibrium with the rest.  From 1 second on they
  evolve separately.  This means that inverse beta decay:
  \begin{equation}
    \overline\nu_e + p \to n + e^+
  \end{equation}
  stops turning protons back into neutrons.  But a neutron can
  spontaneously decay to a proton, with 10 percent decaying every 100
  seconds.  So the ratio of protons to neutrons starts growing.

  Also during this epoch the available thermal energy falls below that
  required to create electron-positron pairs and so positrons pretty
  much disappear. At the same time, thermal energies start to fall
  below the binding energies of small nuclei, most notably the
  deuteron.  This means that protons and neutrons can start coalescing
  to form helium and lithium nuclei without being blasted apart by
  high energy photons. By the end of this period, essentially all the
  neutrons are bound into light nuclei. The ratio of protons to
  neutrons and the relative abundances of primordial elements are
  fixed at this time.

\item[400 s to $10^5$ years:] The universe consists of a plasma of
  protons, helium nuclei, electrons and photons, and decoupled
  neutrinos. Any remaining free neutrons have long since decayed away.
  The photons interact easily with the free electrons via the Compton
  effect and thermal equilibrium between radiation and matter is
  maintained.

\item[$10^5$ years to the present:] At last it is cool enough that
  electrons and nuclei can combine to form atoms, since the photon
  energies are too low to easily ionize the atoms.  Later the photon
  energies fall below even that level necessary to excite atomic
  transitions, so there is no longer any efficient way for photons to
  transfer energy.  The universe becomes transparent to light as the
  photons decouple from the matter. Those primordial photons are still
  wandering around the universe.
\end{description}


\section{Observational Tests}
\label{sec:observational_tests}

Three pieces of observational evidence can serve as a testing ground
for the theories of what went on in the early universe:
\begin{itemize}
\item the discovery of the $3\units{K}$ microwave background radiation,
\item the measurement of the ratio of photons to baryons,
\item the observation of the relative abundances of hydrogen, helium,
  deuterium, and lithium in the interstellar medium.
\end{itemize}
Let's look at each in turn.

Penzias and Wilson's discovery of the $3\units{K}$ microwave radiation in 1965
was a stunning confirmation of the Big Bang model.  This radiation
seems to be coming from everywhere in the universe.  It has a
{\em blackbody} spectrum corresponding to a body at $3\units{K}$.  A simple
calculation shows that these photons are the remnants of the Big Bang,
the photons that decoupled at $\sim 10^5$ years and have been cooling ever
since!

The present observed ratio of photons to baryons seems to be about
$10^9$. Recall our previous claim that these photons are those
produced during the annihilation of almost equal numbers of hadronic
particles and antiparticles. If Grand Unified Theories (GUTs) are
correct in their description of the $X$-boson, then the slightly
different decay modes for $\overline X$ and $X$ should lead to a quark
excess (over antiquarks) of about 1 in $10^9$.  Various versions of
GUTs are now being developed, and each version will presumably lead
to an estimate of the $X$-$\overline X$ decay asymmetry.  See
Problem~
%\ref{chapter:cosmology}.
\ref{prob:X_asymmetry}.  The 1 in
$10^9$ figure thus provides an ``experimental'' test of various
possible grand unified theories.  It is known so far that the simplest
conceivable scheme for {\em grand unification} (so-called minimal
coupling) {\em does not} get the $X$ and $\overline X$ decay modes
right, and thus is effectively ruled out as a viable theory on
observational grounds.

Finally let's consider the question of relative abundances of the
light elements.  The more neutrons there were at the end of the time
of nuclear synthesis, the higher the ratio of helium to hydrogen would
be. (Ignore for the moment all the other elements; these constitute
less than 1\% of the mass of the universe.)  For example, if there were
seven times as many protons as neutrons, then in a group of sixteen
baryons, fourteen would be protons and two would be neutrons. After
combination (see sketch below), there is one helium nucleus (mass 4)
and twelve hydrogen nuclei (total mass 12).  The universe would then
be 25\% helium, by weight.  See Fig.~\ref{fig:helium_abundance}.

\begin{figure}[tbp]
\begin{center}
\includegraphics[width=3.5in]{cosmology/helium_abundance}
\caption{Fourteen protons and two neutrons combine to form one
  $^4$He nucleus, with 12 protons left over.}
\label{fig:helium_abundance}
\end{center}
\end{figure}

One of the things the exact proton to neutron ratio depends on is the
number of different types of neutrinos that exist.  The more types
there are, the longer the neutrinos could initiate proton-to-neutron
conversions (inverse beta decay) in the period just before nuclear
synthesis, and the more neutrons there would be.  This in turn would
lead to a higher helium abundance in the present universe.  Using
standard nuclear physics, the helium abundances for two, three, or
four neutrino types can be calculated and compared to the
observational abundance data.  It turns out that three neutrino types
is the best fit. This means that cosmology can tell us what no
theoretical arguments have yet been able to show, that there are just
three neutrino types and thus just three generations of quarks and
leptons!  The predictions have been confirmed recently by measuring
the uncertainty in the mass of the $Z^0$ boson.

\newpage

\section*{Problems}
\label{sec:cosmology_problems}
\markright{PROBLEMS}

\begin{problem}
Use the time in Table~\ref{table:early_universe} to calculate
  the rest energy associated with the end of Grand Unification.  What
  particles would have about this mass?
\label{prob:grand_unification}
\end{problem}

\begin{problem}
About how long after the Big Bang would there be insufficient
  energy to create $\Lambda$-$\overline\Lambda$ pairs?  The rest
  energy of a $\Lambda$ is $1116\units{MeV}$.
\label{prob:Lambda_pair_production}
\end{problem}

\begin{problem}
 Assuming that one photon is released in each baryon-antibaryon
  annihilation, why are there $10^9$ photons/baryon after annihilation
  and not some other number?
\label{prob:baryon_abundance}
\end{problem}

\begin{problem}
What is the thermal energy per particle available at $t = 1\units{s}$?  How does this
compare with the combined rest energies of an electron-positron
pair?  What happens to positrons after this time?
\label{prob:electron_pair_production}
\end{problem}

\begin{problem}
List the types of particles you might expect to find abundantly
in the early universe during the following epochs:
  \begin{enumerate}
  \item $10^{-10}$ s to $10^{-4}$ s after the Big Bang
  \item 1 s to 400 s after the Big Bang
  \item $10^5$ years after the Big Bang
  \end{enumerate}
\label{prob:particles_and_epochs}
\end{problem}

\begin{problem}
What particle energy is associated with the time of quark
confinement?  What hadron has about this rest energy?
\label{prob:quark_confinement}
\end{problem}

\begin{problem}
Use Eq.~(\ref{eq:early_univ_temp}) to estimate the photon
temperature at the present time ($\sim$10--15 billion years after
the Bang).  Note that this is an overestimate since the assumptions
of Eq.~(\ref{eq:early_univ_temp}) have broken down.
\label{prob:background_radiation}
\end{problem}

\begin{problem}
Transition energies for typical atomic states are about
1--$10\units{eV}$. That is, photons in this energy range are captured 
by atoms to excite upward transitions.  Use Eq.~(\ref{eq:early_univ_energy}) 
to estimate when in the early universe this ``capture effect'' became
energetically impossible.  
\label{prob:transparent_universe}
\end{problem}

\newpage

\begin{problem}
The $X$ and $\overline X$ bosons decay slightly differently
  because of a small flaw in charge conjugation invariance.  Suppose
  the decay scheme for $X$ had the following branching ratios:
\begin{center}
\includegraphics[width=4cm]{cosmology/X_decay}
\end{center}
  That is, there is a 90\% probability that $X$ decays to two quarks,
  and a 10\% chance that $X$ decays to an antiquark plus an antilepton.
  Similarly suppose $\overline X$ decays by
\begin{center}
\includegraphics[width=4.2cm]{cosmology/antiX_decay}
\end{center}

  \begin{enumerate}
  \item Take 3000 random $X$'s and 3000 random $\overline X$'s.  How
    many each of quarks, leptons, antiquarks, and antileptons would
    appear after all $X$'s and $\overline X$'s decayed?
  \item Let all the quarks combine into baryons, and all the
    antiquarks combine into antibaryons.  How many of each are there?
  \item Now let all the particles and antiparticles annihilate as
    completely as possible.  What's left over?  Assume one photon is
    released from each annihilation.
  \item Calculate the photon to baryon ratio for this ($X$, $\overline
    X$) decay scheme.
  \end{enumerate}
  \label{prob:X_asymmetry}
\end{problem}

%%%%%%%%%%%%%%%%%%%%%%%%%%%%%%%%%%%%%%%%%%%%%%%%%%%%%%%%%%%%%%%%%%%%%%%


