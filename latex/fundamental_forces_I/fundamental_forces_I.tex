

%%%%%%%%%%%%%%%%%%%%%%%%%%%%%%%%%%%%%%%%%%%%%%%%%%%%%%%%%%%%%%%%%%%%%%%

\chapter[Fund.\ Forces and Interactions]{Fundamental Forces and Interactions}
%\chapter{Symmetry, Quarks and Color}
\label{chapter:quarks}
%\setcounter{ex}{0}

\section{Introduction}
\label{sec:quarks_intro}

If the universe is like a movie, then the elementary particles
that we discussed in the previous chapter are like the cast for
this movie.  But what if actors in a movie didn't interact at all
with each other or with anything else in the movie?  Imagine that
you went to see {\it Star Wars:  The Rise of Skywalker}.
The movie started and you saw Daisy Ridley, John Boyega, 
Adam Driver, a CGI version of Carrie Fisher and a cute, beach-ball-like android 
on the screen, but they didn't do anything.  They
didn't interact at all with each other.  They didn't talk with
each other.  They didn't fly cool spaceships from one planet to
another or engage in long fight scenes with conveniently-colored
light sabers.  They didn't fight against an evil Emperor who {\bf supposedly}
had died in a previous trilogy and for some ridiculous reason was
back again.  And they didn't inexplicably kiss each other at the
end of the movie. Nothing at all happened -- they just sat there.   An
interesting movie it would not be.\footnote{Come to think of it, that probably
would have been better than the actual movie, which was {\bf so}
bad that it was actually quite entertaining to see how bad it could
be.}

That's what the universe would be like without any forces or
interactions.  Only worse.  A lot worse.  The universe would be a
really boring place without any interactions.  Nothing of any
interest would happen.  In fact, it would be {\bf so} boring that
there wouldn't even be anyone to realize just how bored they were.

Forces probably rank \#1 on the list of things that we most take
for granted.  Particles are all very nice, but if they don't
interact with anything else, then for all intents and purposes
they might as well not even exist.\footnote{In fact, you could
have an interesting debate as to what it really {\it means} for
something to exist, but we won't go into that here.} We would
never be able to detect the effects of any particle that didn't
experience at least {\it some} kind of force/interaction.

In this chapter, we discuss our current understanding of the
fundamental forces of the universe.  We describe the four
fundamental forces -- strong, electromagnetic, weak, and
gravitational -- and discuss efforts being made by physicists to
describe all of these forces under the framework of a single,
unified theory.  We then discuss Richard Feynman's quantum field
theory that describes {\it how} forces might actually {\it work},
and apply these approach to the electromagnetic and strong forces.
(We will extend the approach to the weak force in the next
chapter.)

% Full circle -- back in September, forces.

% Unification.

\section{The Four Fundamental Forces of Nature}
\label{sec:four_fundamental_forces}

Throughout the year, we have discussed a wide variety of forces:
tension forces, normal forces, friction forces, air and fluid
drag, spring forces, gravitational forces, electric and magnetic
forces, etc.  But most of these forces are just different
manifestations of the same fundamental force.  For instance, a
normal force between two objects ``in contact with each
other''\footnote{We put ``in contact with each other'' in quotes
because nothing ever really {\it touches} anything else on a
subatomic level.  The sensation or appearance of ``contact'' is
the result of the repulsive force becoming very large.} is just
electrostatic repulsion between the electrons at the surface of
one object and those at the surface of the other object.

During the past century, physicists have typically talked about
four ``fundamental'' forces:  the strong force, the
electromagnetic force, the weak force, and the gravitational
force.  You are already familiar with the electromagnetic force
(which affects any particle with an electric charge) and
gravitational force (which affects any particle with mass). The
strong force is the force that holds protons and neutrons together
in the nuclei of atoms. This is no small matter here: the
electric force is extremely repulsive between protons packed
together in a nucleus, so there clearly must be a force that is
even stronger pulling them together.  But the strong force is only
active over short distances, those comparable to the sizes of
atomic nuclei. For separations greater than that, the strong
force is ineffective.  As discussed in the previous chapter, the
strong force affects all of the hadrons.  Actually, since the
hadrons are all made up of quarks, it can be said that the strong
force is the force that is associated with quarks.

The weak force is somewhat of a ``miscellaneous/etc'' force; in
fact, all of the elementary particles can experience the weak
force.  For quarks or charged particles, though, the weak force is
subordinate to the strong and/or electromagnetic force(s).  Only
for uncharged leptons (i.e., neutrinos) is the weak force
dominant.  But since the weak force is {\bf so} weak, neutrinos
can pass through other objects with only minimal interactions.
Right now, over $10^7$ neutrinos pass through you every second. In
fact, the vast majority of neutrinos that arrive at the Earth
(from the Sun) pass through the entire planet without interacting
with any other matter!

The weak force is also important in a lot of decay processes. For
example, $\beta$-decay by which a neutron decays ($n \rightarrow p
+ e^- + \nu$) is a weak interaction.  In fact, the weak
interaction was first proposed because this decay process couldn't
be explained with any of the other fundamental interactions.

\section{Quantum Field Theory}
\label{sec:quantum_field_theory}

Modern physical theory seeks to explain all four of the
fundamental interactions --- gravity, electromagnetism, strong,
and weak --- in terms of quantum fields.  Here's how classical
models explain how two particles exert forces on one another: each
particle sets up a {\em field} that the other particle reacts to,
and these fields obey certain equations (like Faraday's Law).
This, of course, isn't really an explanation at all, because it
doesn't really explain what a field is and why one particle should
produce a field and why another particle should respond to this
field.

In quantum field theory, the fields themselves are quantized;
changes in the fields occur in discrete lumps.  The lumps of
energy and momentum can be considered particles, often called {\em
messenger particles}.  As an analogy, think of two people playing
catch.  The ball passes back and forth between the players and
transmits the exchange of momentum and energy.

Let's see how this works for electric forces.  The quantum field
theory for electromagnetic interactions is called quantum
electrodynamics or QED.  In this theory an electrically charged
particle --- an electron, for instance --- makes its presence
known not by setting up an electric field, but by continually
emitting and absorbing electromagnetic messenger particles, or
{\em photons}. Because these photons are not detected as particles
of light, but serve only to transmit energy and momentum between
particles, they are also called {\em virtual photons}.  In fact,
at any given time, an electron is literally surrounded by a cloud
of these virtual photons, appearing and disappearing.

Where does the energy come from to create all these photons?
Doesn't energy have to be conserved?  The answer is that
short-term violations of the law of conservation of energy are
permitted by the quantum uncertainty relations, as we have already
seen in tunneling.  The uncertainty relation for time and energy,
analogous to $\Delta x\, \Delta p \geq \hbar/2$, is
\begin{equation}
\Delta E\, \Delta t \approx \hbar \label{eq:energy_time}
\end{equation}
In effect, Eq.~(\ref{eq:energy_time}) permits particles to borrow
an amount of energy $\Delta E$ from the surrounding space so long
as the energy is returned within a time $\Delta t \approx
\hbar/\Delta E$. In this view, even empty space itself is loaded
with activity, as virtual particles and messengers pop into
existence and then quickly disappear!\footnote{Experiments have
measured these effects in ``empty'' space. Specifically, even a
well-shielded vacuum is observed to have electromagnetic ``noise''
due to the blipping into and out of existence of these virtual
photons.}

Current theory holds that there are messenger particles like
photons that act as the intermediaries for all of the fundamental
interactions.  For the weak interaction, they are the massive (as
in, having mass) {\em vector bosons} $W^+$, $W^-$, and $Z^0$.  For
the strong interaction, they are the colored {\em gluons}, which
act between quarks.  The properties of the messenger particles are listed 
in Table~\ref{table:messengers}.  You'll learn more about these messenger
particles in this and the next chapter. In addition, a particle
called the {\em graviton} is postulated as the messenger for
gravity.

\begin{table}[tbp]
\caption{The messenger particles. \label{table:messengers}}
\begin{tabular}[t]{cccccc}
  Symbol & Name & Mass (MeV/$c^2$) & Spin & Charge & Interaction \\
  \hline\hline
  $\gamma$ & photon & 0 & 1 & 0 & Electromagnetic \\
  $W^-$ & double-you & 80,400 & 1 & -1 & Weak \\
  $Z^0$ & zee-zero & 91,200 & 1 & 0 & Weak \\
  gl & gluon & 0 & 1 & 0 & Strong \\
  g & graviton? & 0 & 2 & 0 & Gravity \\
  \hline
\end{tabular}\\[0.5ex]
Note: the antiparticle of $W^-$ is $W^+$, and the antiparticle of
a green-antired gluon is a red-antigreen gluon, etc.
\end{table}

\section{Feynman Diagrams}
\label{sec:feynman_diagrams}
\label{section:feynman}

In putting the ideas of quantum field theory together in the
1940's, Richard Feynman developed a method of diagramming
interactions. These diagrams can actually be translated directly
to equations using a set of Feynman rules that associates
different parts of the diagram to various factors and terms in the
equations.  We will not try to do that, but we will use
Feynman-like diagrams to sketch reactions.

Let's diagram the interaction of two electrons repelling each
other via the electromagnetic interaction.  We've just seen that
one electron emits a virtual photon as the messenger, and the
other electron receives it.  If we plot time upward and space
horizontally (like a space-time diagram) we find the interaction
looks like Fig.~\ref{fig:electron_photon}.


\begin{figure}[b]
\begin{center}
\includegraphics[width=1.6in]{fundamental_forces_I/electron_photon}
\caption{Feynman diagram of two electrons interacting
electromagnetically by the exchange of a photon.}
\label{fig:electron_photon}
\end{center}
\end{figure}

The particles that interact are drawn as solid straight lines,
indicating constant velocity before and after the interaction. The
lines can change direction only at a vertex, the point where two
solid lines and a squiggly photon line meet.  The vertex
represents the emission or absorption of the messenger particle,
events which can change the properties of the interacting
particles.  In this case, the momentum of each electron changes at
the vertex.

The weak and color interactions can also be illustrated in this
way. The messenger particles for the weak interactions are the
$W^+$, $W^-$, and $Z^0$ bosons, which on a diagram are represented
by closely spaced parallel lines.  For the strong interaction
between the quarks, the messengers are colored gluons, which
appear as double squiggly curves.  Figs.~\ref{fig:muon_W} and
\ref{fig:quark_gluon} show examples of weak and strong
interactions.

\begin{figure}[tbp]
\begin{minipage}[t]{6.0cm}
\begin{center}
\includegraphics[width=4.5cm]{fundamental_forces_I/muon_W}
\caption{Feynman diagram showing a muon decaying by the weak
interaction.} \label{fig:muon_W}
\end{center}
\end{minipage}
\hfill
\begin{minipage}[t]{6.0cm}
\begin{center}
\includegraphics[width=5cm]{fundamental_forces_I/quark_gluon}
\caption{Feynman diagram showing quarks interacting strongly by
exchanging a gluon.} \label{fig:quark_gluon}
\end{center}
\end{minipage}
\end{figure}

\section{Quark Color}
\label{sec:quark_color}

At first glance, the quark model of the hadrons as outlined by
Gell-Mann is very appealing in that it seems to explain the occurrence
of most of the observed particles.  But several problems come forward
when a closer look is taken.  Consider the three up quarks that make
up the $\Delta^{++}$ on page~\pageref{ex:delta} in 
Chapter~\ref{chapter:particles}.  The $\Delta^{++}$ is the lowest
mass doubly-charged baryon, so presumably the quarks are each in the
ground state.\footnote{That is, there are other, higher mass baryons
with $+2$ charge that are also made up of three up quarks, but these
correspond to excited state configurations of the quarks.}  But quarks
are fermions (spin 1/2) and so the Pauli exclusion principle should
apply!  How can three identical quarks all occupy the same state?

The way around this problem is to postulate that the three up quarks
in $\Delta^{++}$ (or the three quarks in any baryon) are not
identical, but are each a different ``color.''  Now of course they
don't actually appear colored, but for the reasons described below we
call this extra property of quarks color, and a quark can be red ($R$)
or blue ($B$) or green ($G$).  The antiquarks come in anti-colors:
anti-red ($\overline R$), anti-blue ($\overline B$), or anti-green
($\overline G$).  No kidding, we really use these labels.

There is a single simple rule about quark color and here it is:

\boxittext{
  {\bf The Colorless Rule:} No physically observable particle ever
  carries net color.
}

By never carrying a net color, we mean that the colors of the
constituent quarks in a particle must cancel out, so that the
composite particle is colorless.  This can happen in three ways.
\begin{itemize}
\item[1.] A particle can contain a colored quark and an anti-quark of
  opposite color, i.e., $R\overline R$, $B\overline B$, or $G\overline G$.
\item[2.] A particle can contain three quarks, one of each color
  (i.e., $RBG$).  The colors cancel out in much the same way as the
  three primary colors of light mix to give white (colorless) light.
  This is the reason we say quarks carry color rather than using some
  other name.
\item[3.] A particle can contain three antiquarks, one of each
  anti-color (i.e., $\overline R\overline B\overline G$).
\end{itemize}

The colorless rule automatically picks out those combinations of
quarks that actually occur in nature.  We never see a lone quark, or
combinations such as $qq$, $q\overline q\overline q$, or $qqq\overline
q$, because these would of necessity show net color.  Furthermore, the
rule automatically forbids the possibility of observing particles with
fractional baryon number or fractional charge.  The Rule is crafted so
that we are only allowed to observe what we in fact do observe.

The colorless rule is actually a simplified way of describing a very
deep symmetry in nature, {\em color symmetry}.  Since net color can
never be observed, it really can't matter whether we label the quarks
in a proton as $RBG$ or $RGB$ or $BGR$ or $BRG$ or $GRB$ or $GBR$.
We'll see in the next section how the colorless rule is enforced by
the exchange of {\em gluons}, the messenger particles that actually
hold the quarks together and keep them confined inside hadrons.

\section[Chromodynamics]{Chromodynamics: The Theory of Colored Quarks}
\label{sec:chromodynamics}

Let's now investigate the strong interaction between quarks.  The
strong interaction is actually a manifestation of the color property
of quarks. Just as with electric charges, color charges repel if alike
and attract if different.  Now a quark, like any particle of quantum
field theory, continually emits and absorbs messenger particles and is
thus surrounded by a small cloud of virtual quarks and gluons. Nearby
quarks then detect the presence of the first quark by interacting with
its cloud of colored gluons.

\begin{figure}[tbp]
\begin{center}
\begin{minipage}{10cm}
\begin{center}
\includegraphics[width=6cm]{fundamental_forces_I/quark_color_change}
\caption{Feynman diagram showing a quark changing color as it emits a
  gluon.}
\label{fig:quark_color_change}
\end{center}
\end{minipage}
\end{center}
\end{figure}

It is important to understand what happens to a quark when emitting or
absorbing gluons: {\em as a quark emits a gluon, its color changes}.
This color change can be seen in Fig.~\ref{fig:quark_color_change}.
Note that a gluon carries both a color and an anticolor.  In the
example, the red up quark on the left emits a red-antiblue gluon and
changes into a blue up quark. You can think of this by saying that the
red color leaves the quark and goes into the gluon.  The antiblue
color of the gluon is also withdrawn from the quark, leaving the quark
blue. The gluon may then carry color to another quark, or it may
return to the original quark to be re-absorbed.

An important aspect of the color force between quarks is how its
interaction strength varies with distance.  We can best understand
this concept through an analogy with electric forces.  As you recall
from our study of Gauss' Law, the electric flux through a surface
depends only on the {\em net charge} inside the enclosed volume.
Similarly, the color field depends only on the {\em net color} inside
a given volume.  We must thus consider the effect of the gluon
messenger particles, since they can carry the color charge away from a
quark.

The color carried by a gluon leads to the effect called
{\em camouflage}. Consider a red quark.  Suppose while you are trying to
test the color inside a certain region surrounding the quark, it emits
a red-antigreen gluon, leaving behind a green quark.  The gluon can't
go far without violating the uncertainty principle and unless it finds
another quark, it soon returns and is re-absorbed.  But during the time
it's away, the net color in a large region enclosing the fleeing gluon
is still red, while in a smaller region missing the gluon, the net
color of the quark looks green. The combined effect of gluons
continually leaving and returning is that the closer you get to the
red, the less red it appears.  Thus the effective color charge
decreases as you move closer to it.

Detailed calculations show that this camouflage effect makes the color
charge continually decrease at smaller and smaller separations.  In
fact, it turns out that on distance scales smaller than a proton
radius, the effective quark color charge is so diminished that the
three different colored quarks inside a proton hardly affect each
other and move almost independently. The gluon force between quarks
thus acts like a rubber band tied between two balls.  The balls move
freely until the rubber band gets stretched.  Then the more they try to
move apart, the harder the rubber band force pulls them back together.
Continuing the rubber-band analogy, you can see that, as you try to
separate quarks, the effective color charge increases rapidly and the
inter quark forces become huge.  In fact, the energy needed to pull
apart quarks is so great that a quark-antiquark pair is created
instead, as shown in Fig.~\ref{fig:no_free_quark}. Instead of a free
quark, we just create a new meson.  This is why we've never observed
free quarks; all quarks are permanently confined inside hadrons.

\begin{figure}[tbp]
\begin{center}
\includegraphics[width=4in]{fundamental_forces_I/no_free_quark}
\caption{Trying to pull a quark out of a baryon fails.  The energy
  creates a new meson instead.}
\label{fig:no_free_quark}
\end{center}
\end{figure}

\section{Interactions and Messenger Bosons}
\label{sec:interactions}

\begin{figure}[!b]
\begin{center}
\begin{minipage}{10cm}
\begin{center}
\includegraphics[width=6cm]{fundamental_forces_I/feynman_electron_photon}
\caption{Feynman diagrams of typical electromagnetic interactions.
An electron changes momentum by (a) emitting a photon ($e^- \rightarrow
e^- + \gamma$), or (b) absorbing a photon ($e^- + \gamma \rightarrow e^-$).} 
\label{fig:feynman_electron_photon}
\end{center}
\end{minipage}
\end{center}
\end{figure}


Before describing the weak interaction on a fundamental level,
let's review how photons and gluons serve as the messengers for
the electromagnetic and strong interactions between leptons and
quarks. Messenger particles that transmit interactions, such as
photons, gluons, and the $W$'s and $Z^0$'s, are also called
intermediate bosons, because they mediate interactions and carry
integer spin.

Consider first the most familiar lepton, an electron, emitting or
absorbing a photon.  The basic interaction occurs at a vertex in a
Feynman diagram.  See Fig.~\ref{fig:feynman_electron_photon}.  As
the electron emits or absorbs a photon, its energy and momentum
change, but it remains an electron.

Turn now to the description of a quark emitting or absorbing a
gluon. See Fig.~\ref{fig:feynman_quark_gluon}.  Notice that only
gluons carrying red ($R\overline R$, $R\overline B$, or
$R\overline G$) can be emitted by red quarks, while only gluons
carrying antired ($R\overline R$, $B\overline R$, or $G\overline
R$) can be absorbed by red quarks.  Similar rules apply for the
gluons emitted or absorbed by blue or green quarks.  As the quarks
in the diagram emit or absorb a gluon, their energy, momentum, and
color may all change, but the {\em flavor} remains the same, i.e.,
a down quark remains a down quark, an up quark remains an up
quark, etc.


\begin{figure}[tbp]
\begin{center}
\begin{minipage}{10cm}
\begin{center}
\includegraphics[width=7cm]{fundamental_forces_I/feynman_quark_gluon}
\caption{Feynman diagrams of typical quark-gluon interactions.  An
up quark changes color by (a) emitting a gluon ($Ru \rightarrow
Bu + R\bar{B}gl$), or (b) absorbing a gluon ($Ru + B\bar{R}gl \rightarrow
Bu$).}
\label{fig:feynman_quark_gluon}
\end{center}
\end{minipage}
\end{center}
\end{figure}

The weak interaction, mediated by the $W^\pm$ and $Z^0$ bosons, is
probably the hardest one to understand.  You saw in
Chapter~\ref{chapter:particles} that the weak interaction violates
strange\-ness conservation, and it is known that certain other
symmetries are violated as well.  We know that both quarks and
leptons seem to ``feel'' the weak interaction, but it affects
quarks and leptons somewhat differently.  Since the $W^\pm$ bosons
carry electric charge, the particle emitting or absorbing a $W$
boson must change its identity in a major way.


Consider first how $W$ bosons interact with quarks.  A $W$ boson
{\em changes a quark's flavor}.  That is, if an up quark emits or
absorbs a $W$, then it is no longer an up quark.  In
Fig.~\ref{fig:feynman_quark_W}, we see a red strange quark
emitting or absorbing a W and becoming a red up quark.  Its
energy, momentum, charge and flavor change, but the quark color
remains red.

While Fig.~\ref{fig:feynman_quark_W} shows typical interactions at
a quark-$W$ vertex, we can derive several other versions based on
the symmetry properties of Feynman diagrams.  Look at
Table~\ref{table:weak_quarks}.  Each operation is based on the
standard version.  Also, any {\em combination} of these operations
yields a valid quark-$W$ interaction.

\begin{figure}[!b]
\begin{center}
\begin{minipage}{10cm}
\begin{center}
\includegraphics[width=7cm]{fundamental_forces_I/feynman_quark_W}
\caption{Feynman diagrams of typical quark-$W$ boson interactions.
A
  strange quark changes flavor by (a) emitting a $W^-$ ($Rs \rightarrow
Ru + W^-$), or (b) absorbing a $W^+$ ($Rs + W^+ \rightarrow Ru$).}
\label{fig:feynman_quark_W}
\end{center}
\end{minipage}
\end{center}
\end{figure}


\begin{table}[!b]
\begin{center}
\caption{Weak interactions of quarks.} \label{table:weak_quarks}
\begin{tabular}[tbp]{lc}
\hline
standard version & $s \to u + W^-$ \\[0.5ex]
reverse arrow    & $u + W^- \to s$ \\[0.5ex]
particle out $=$ anti-particle in & $\overline u + s \to W^-$ \\[0.5ex]
change to antiparticles & $\overline s \to \overline u + W^+$ \\[0.5ex]
replace $s$ with $d$ & $d \to u + W^-$ \\
\hline
\end{tabular}
\end{center}
\end{table}

\newpage

Finally, let's look at how $W$'s interact with leptons.  A $W$
boson {\em changes a lepton's ``family membership.''}  We saw
earlier that there are three types or families of leptons:
electron, muon, and tauon.  Each family contains a negative
particle, a positive antiparticle, and a neutrino and
antineutrino.  The basic interaction is that a lepton emits a
$W^+$ or $W^-$ and changes into a different lepton of the same
family.  This process of changing to a different lepton of the
same family is what we mean by the phrase ``change of family
membership.''  As Fig.~\ref{fig:feynman_lepton_W} illustrates, the
electron emits a $W^-$ or absorbs a $W^+$ and changes its charge
and family membership, becoming an electron neutrino.  But, it
maintains its lepton number, $L_e = +1$.

As with the quark interactions, we can derive several versions of
the lepton-$W$ interaction.  These are listed in
Table~\ref{table:weak_leptons}.  Each operation is based on the
standard version.  Again, any combination of these operations
yields a valid lepton-$W$ interaction.


\begin{figure}[!t]
\begin{center}
\begin{minipage}{10cm}
\begin{center}
\includegraphics[width=6cm]{fundamental_forces_I/feynman_lepton_W}
\caption{Feynman diagrams of typical lepton-$W$ boson
interactions.  An electron changes its family membership by 
(a) emitting a $W^-$ ($e^- \rightarrow \nu_e + W^-$), or
(b) absorbing a $W^+$ ($e^- + W^+ \rightarrow \nu_e$).}
\label{fig:feynman_lepton_W}
\end{center}
\end{minipage}
\end{center}
\end{figure}

\begin{table}[!t]
\begin{center}
\caption{Weak interactions of leptons.} \label{table:weak_leptons}
\begin{tabular}[tbp]{lc}
\hline
standard version & $e^- \to \nu_e + W^-$ \\[0.5ex]
reverse arrow  & $\nu_e + W^- \to e^-$ \\[0.5ex]
particle in $=$ antiparticle out & $\overline\nu_e + e^- \to W^-$ \\[0.5ex]
change to antiparticles & $e^+ \to \overline\nu_e + W^+$\\[0.5ex]
replace $e$ with $\mu$ or $\tau$ & $\mu^- \to \nu_\mu + W^-$\\[0.5ex]
\hline
\end{tabular}
\end{center}
\end{table}

% \newpage

An important thing to note here is the equivalence in these
processes and in the diagrams between a particle ``going forward
in time'' and an antiparticle ``going backward in
time.''\footnote{As an example of this seemingly-crazy idea, try
drawing a Feynman diagram for the annihilation of an electron and
positron, resulting in the production of two photons.
Mathematically, you could describe the same process as an electron
moving along, emitting two photons, and then going backwards in
time.  (Or, alternately, an electron emits a photon, gets
deflected, then emits another photon, and then goes backwards in
time.)
%See Problems~\protect\ref{chapter:interactions}.\protect\ref{problem:gumby_quark_W}
%and
%\protect\ref{chapter:interactions}.\protect\ref{problem:gumby_lepton_W}.
Note that this does not mean that quantum theory predicts the
possibility of macroscopic objects traveling backwards in time ---
it does not.} In Fig.~\ref{fig:feynman_lepton_W}, for instance, an
electron is changed into a $\nu_e$ neutrino either by emitting a
$W^-$ or by absorbing a $W^+$.  This can result in some ambiguity:
a process in which particle $A$ emits a $W^-$, which is then
absorbed by particle $B$, could just as easily be re-written as a
process in which particle $B$ emits a $W^+$, which is then
absorbed by particle $A$.

\newpage

\section*{Problems}
\label{sec:quarks_problems}
\markright{PROBLEMS}

\begin{problem}
Think about what you are doing (or not doing) right at this
minute. (a) Identify (and write down) as many forces as you can
that are acting on you right now and/or are affecting your
environment. (b) Describe what your life would be like if there
were no forces or interactions at all. 
\label{prob:think_about_forces}
\end{problem}

\begin{problem}
(a) What is the minimum energy fluctuation $\Delta E$
required  to create a virtual electron-positron pair?  
(b) Roughly how long could such a virtual pair survive?
\label{prob:pair_production}
\end{problem}

\begin{problem} 
The $Z$-boson (one of the mediators of the weak force) has a
  mass of $91.2\units{GeV/$c^2$}$.
  \begin{enumerate}
  \item What is the minimum energy fluctuation required to create a
    virtual $Z$-boson?
  \item Roughly how long could such a virtual $Z$ survive?
  \item What is the farthest that this particle could travel during
    this lifetime?
  \item How does this result compare with the known range of the weak
    force ($\sim 10^{-3}\units{fm}$)?
  \end{enumerate}
\label{prob:Z_boson}
\end{problem}

\begin{problem}
Several years ago, a group of physicists claimed to have
discovered a fifth force with a range of approximately $100\units{m}$.  
If this force had turned out to be valid (few people
believe this to be the case), what would have been the approximate
(order of magnitude) mass of its messenger particle?  Express your
answer in eV/$c^2$.  
\label{prob:fifth_force}
\end{problem}

\begin{problem}
Draw a Feynman diagram corresponding to the annihilation of
an
  electron-positron pair.  An electron moves to the right and hits a
  positron moving to the left.  They annihilate each other, resulting
  in the formation of two photons, one moving left and the other
  moving right.  {\em Hint:} You will need two vertices.
\label{prob:pair_annihilation_diagram}
\end{problem}

\begin{problem}
A Feynman diagram can be played with like a Gumby doll.  You
can bend all the branches up and down, as long as you follow one simple
rule: every time you bend an upward-going branch downward (or a
downward-going branch upward), you replace the particle with its
antiparticle.  For example, in Fig.~\ref{fig:muon_W}, if the
$\nu_\mu$ branch is bent downward, you can get the interaction
$\overline\nu_\mu + \mu^- \to \overline\nu_e + e^-$.
  \begin{enumerate}
  \item Verify that this new reaction satisfies all the appropriate
    conservation laws.
  \item Use the Gumby approach to come up with several other
    possibilities for interactions, starting with Fig.~\ref{fig:muon_W}.
  \end{enumerate}
\label{prob:gumby}
\end{problem}

\begin{problem}
Show with some examples that the colorless rule yields particles
  with integer charge.
\label{prob:colorless_rule}
\end{problem}

\begin{problem}
An antiblue antiquark emits a green-antiblue gluon.  What color
  is the antiquark after emission?
\label{prob:quark_color}
\end{problem}

\begin{problem}
Explain in your own words how {\em camouflage} leads to
confinement of quarks inside hadrons.
\label{prob:camouflage}
\end{problem}

\begin{problem} 
Draw a diagram like Fig.~\ref{fig:feynman_quark_W} for each of
the reactions in Table~\ref{table:weak_quarks}.
\label{problem:gumby_quark_W}
\label{prob:gumby_ii}
\end{problem}

\begin{problem}
Draw a diagram like Fig.~\ref{fig:feynman_lepton_W} for each of
the reactions in Table~\ref{table:weak_leptons}. 
\label{problem:gumby_lepton_W}
\label{prob:gumby_iii}
\end{problem}

\begin{problem} 
For each vertex diagram below, identify the missing particles,
colors, or flavors denoted by a question mark.  In other words,
wherever you see a `?', replace it with a type of particle, a color,
or a type of quark.\\
\includegraphics[width=4in]{fundamental_forces_I/missing_particles_1}
\label{prob:identify_missing_property}
\end{problem}
%%%%%%%%%%%%%%%%%%%%%%%%%%%%%%%%%%%%%%%%%%%%%%%%%%%%%%%%%%%%%%%%%%%%%%

