%%%%%%%%%%%%%%%%%%%%%%%%%%%%%%%%%%%%%%%%%%%%%%%%%%%%%%%%%%%%%%%%%%%%%%

\chapter[Fund.\ Forces and Interactions II]{Fundamental Forces and Interactions II}
\label{chapter:interactions}
%\setcounter{ex}{0}

\section{Introduction}
\label{sec:interactions_intro}

%We are now ready to study particle interactions at the most
%fundamental level of quarks and leptons.  You saw in
%Section~\ref{sec:feynman_diagrams} that photons carry the electromagnetic
%interaction between charged particles, thus changing their energy and
%momentum.  In this chapter you'll learn how {\em gluons} carry the
%strong interaction between quarks, thus changing their color and how
%the $W^\pm$ and $Z^0$ bosons carry the weak interaction among quarks
%and leptons, thus changing a quark's flavor or a lepton's family
%membership.

This chapter continues from the discussion begun in the previous
chapter.  We now have the tools that we need to study particle
interactions at the most fundamental level of quarks and leptons.
In this chapter, you'll learn how to sketch particle reactions to
reveal the details of the fundamental interactions.  And you'll
learn how to predict a particle's decay scheme and lifetime from
its mass and internal structure.  We'll look at the three
generations of quarks and leptons and make a case for {\em grand
unified theories} (GUTs).  And finally, we'll return to the early
universe as the best laboratory for testing some of these ideas.


\section[Feynman diagrams]{Feynman Diagrams and Particle Interactions}
\label{sec:feynman_diagrams_interactions}

We now have all the pieces that we need to produce diagrams
corresponding to the particle interactions that we discussed in the
previous class.  All of the conservation laws that were used to
analyze those interactions will be incorporated into these
{\em interaction diagrams}.

In the previous section, we reviewed the key features of Feynman
diagrams as they pertain to the various interactions.  A quark or
lepton goes along, then emits a messenger particle, changing the
properties of the original particle.  Another particle receives and
absorbs the messenger, and the interaction is complete.  Or perhaps
the messenger decays into a quark-antiquark or lepton-antilepton pair.

We can expand the Feynman diagram idea to composite particles like
hadrons by just ``carrying along'' the non-interacting constituents.
Here's the key point: in every diagram, you should consider very
carefully each vertex where a messenger is emitted or absorbed.  Every
vertex must conserve both charge and color.  The net charge before the
emission/absorption process is the same as the net charge after the
process.  Similarly, (net color before) = (net color after).  If you
are careful to conserve both charge and color, then your diagrams will
most likely be correct.

\begin{figure}[tbp]
\begin{minipage}{7cm}
\caption{Interaction diagram for neutron decay, a weak interaction.}
\label{fig:neutron_decay}
\end{minipage}
\hfill
\begin{minipage}{4cm}
\includegraphics[width=4cm]{fundamental_forces_II/neutron_decay}
\end{minipage}
\end{figure}

A few examples will best illustrate this.  Consider the beta-decay of
a free neutron, a process governed by the weak interaction:
\begin{equation}
      n \to p + e^- + \overline{\nu_e}  .
\end{equation}
To show the internal structure, simply expand the composite particle
into quarks and antiquarks and let the reaction go.  See
Fig.~\ref{fig:neutron_decay}.  Note the two types of $W$-boson
vertices in the diagram: one changes the flavor of a quark and the
other creates a lepton-antilepton pair.

Take a look at the two vertices in this diagram.  The first is the
vertex corresponding to the emission of the $W^-$ from the down quark,
which is changed to an up quark by the emission.  Ignore the up and
down quark at the left of the diagram and consider only the particles
involved at that vertex.  If you draw a horizontal dotted line, then
everything below the line is what was present before the emission and
everything above the line is what was present after the emission.
Now, consider charge conservation. Charge before: $Q_{\rm before} =
Q_d = -1/3$ (recall that $Q$ refers to the charge of the particle).
Charge after: $Q_{\rm after} = Q_u + Q_{W^-} = +2/3 - 1 = -1/3$, so
this vertex conserves charge.  Note: we won't draw the horizontal
dotted lines anymore.
\begin{quote}
{\bf Checkpoint 1:} convince yourself that charge is also conserved at
the second vertex: the decay of the $W^-$ into the $e^-$ and $\overline\nu_e$.
\end{quote}
Note that color is automatically conserved at both of these vertices,
since the $W$ particles never have any net color, nor do the $e^-$ and
$\overline\nu_e$.


For the strong interaction there are several possibilities.  The
simplest type is an {\em exchange reaction}, an example of which is
shown in Fig.~\ref{fig:pp_reaction}.  The quarks just get jumbled up
in the collision and then rearrange themselves to make up the end
products.  Note that there is still the exchange of a gluon here.
Also, note that color is conserved at the vertices at both ends of the
gluon.  At the first (lower) vertex, the color is initially $R$ (for
the $d$ quark in the proton).  After the vertex, the color is $B$ (for
the $d$ quark) + $R\overline B$ (for the gluon that has just been
emitted).
\begin{quote}
{\bf Checkpoint 2:} convince yourself that color is also conserved at
the second vertex.
\end{quote}

\begin{figure}[tbp]
\begin{minipage}[t]{5.5cm}
\begin{center}
\includegraphics[width=4cm]{fundamental_forces_II/pp_reaction}
\caption{Interaction diagram for $p + p \to \Delta^{++} + n$, a strong
  interaction.}
\label{fig:pp_reaction}
\end{center}
\end{minipage}
\hfill
\begin{minipage}[t]{5.5cm}
\begin{center}
\includegraphics[width=3.7cm]{fundamental_forces_II/pi_p_reaction}
\caption{Interaction diagram for $\pi^- + p \to \Lambda + K^0$, a strong
  interaction.}
\label{fig:pi_p_reaction}
\end{center}
\end{minipage}
\end{figure}

Another possibility is to annihilate and create quark-antiquark pairs,
mediated by a gluon.  Consider the bombardment of a stationary proton
with a high-energy pions, as shown in Fig.~\ref{fig:pi_p_reaction}.
There are several ways in which the colors could be labeled in this
diagram.  Let's assume that the two quarks in the $\pi^-$ before the
reaction have colors $G$ and $\overline G$.  The three quarks in the
$p$ must always have colors $R$, $G$, and $B$, according to the
colorless rule.  The gluon will then have the color combination
$R\overline G$.

A couple of additional points to note about the reaction in
Fig.~\ref{fig:pi_p_reaction}. First, since strange quarks are more
massive than up quarks, considerable kinetic energy (of the $\pi^-$)
must be converted to mass-energy for this reaction to occur.  In fact,
a $\pi^-$ hitting a stationary proton must have a kinetic energy of at
least $770\units{MeV}$ to produce a $\Lambda$ and $K^0$.  Second, the gluons
that mediate the strong interaction cannot change the net {\em flavor}
of the quarks.  Up and antiup are destroyed together; strange and
antistrange are created together.  As a result, the additive
properties like strange\-ness, baryon number and charge are
automatically conserved.

\begin{figure}[tbp]
\begin{minipage}{6cm}
\begin{center}
\includegraphics[width=4cm]{fundamental_forces_II/sigma-minus_decay}
\caption{Reaction diagram for the strong decay $\Sigma^{\ast -} \to
  \Lambda(1116) + \pi^-(135)$.}
\label{fig:sigma-minus_decay}
\end{center}
\end{minipage}
\hfill
\begin{minipage}{6cm}
\begin{center}
\includegraphics[width=4cm]{fundamental_forces_II/weak_sigma-minus_decay}
\caption{Reaction diagram for the weak decay $\Sigma^-(1197) \to n +
  \pi^-$.}
\label{fig:weak_sigma-minus_decay}
\end{center}
\end{minipage}
\end{figure}


A third possibility for the strong interaction involves the decay of
{\em excited states} of quark configurations.  This can occur whenever
the mass of the parent particle is high enough to have energy
available for creating a quark-antiquark pair.  See
Fig.~\ref{fig:sigma-minus_decay} for an example.

Consider the decays of the particles labeled $\Sigma^-(1197)$ and
$\Sigma^{*-}(1387)$ (the numbers in parenthesis are the masses in
MeV/$c^2$).  A proposed decay scheme might be like that shown in
Fig.~\ref{fig:sigma-minus_decay}.  Note that we've shown the
colors on this diagram since it involves emission and decay of a
gluon. However, when we check conservation of energy, we find that
the mass of the products is greater than $1197\units{MeV/$c^2$}$ but less
than $1387\units{MeV/$c^2$}$.  This means the $\Sigma^{*-}(1387)$ has 
enough rest energy to decay this way but the $\Sigma^-(1197)$ does not.
It must find another way.
% (see
% Problem~\ref{chapter:interactions}.\ref{prob:sigma-minus_decay}).

\section{Particle Lifetimes}
\label{sec:particle_lifetimes}

Let's see how the quark model can help us predict particle lifetimes.
The two most important factors are the type of interaction involved
and the amount of energy available.  We'll first illustrate with an
example and then develop some general rules.

In doing
Problem~
%\ref{chapter:interactions}.
\ref{prob:Sigma_minus_decay},
you'll discover that energy considerations severely limit the possible
decay modes for $\Sigma^-(1197)$.  In fact, only $\Sigma^- \to n +
\pi^-$ is allowed.  See Fig.~\ref{fig:weak_sigma-minus_decay}.
There is a flavor change, because a strange quark has to turn into
an up quark, so a $W^-$ boson must be involved, indicating the weak
interaction.  And because the weak interaction is much weaker than the
strong, it acts much more slowly and $\Sigma^-(1197)$ lives a
relatively long time.  Compare the mean lifetimes: $1.5 \times 
10^{-10}\units{s}$
for $\Sigma^-(1197)$ and $2 \times 10^{-23}\units{s}$ for 
$\Sigma^{*-}(1387)$.
The fact that only a weak decay can occur allows $\Sigma^-(1197)$ to
live 10 trillion times longer!

Here are some general rules about decay schemes and lifetimes.
\begin{itemize}
\item A decay proceeds by the strongest interaction allowed:
  strong, then electromagnetic, then weak.  The typical times for these
  interactions are given in Tables~\ref{table:process_decay_times} and
  \ref{table:mass_lifetimes}.

\item Particles can only decay to lighter particles.  The more
  excess energy available, the faster the decay proceeds.

\item  A baryon's decay products must include a baryon.

\item If quark flavor is not conserved, the decay must proceed by the
  weak interaction (because a $W$ or $Z$ boson must be involved).

\item If photons are involved, the decay process proceeds by the
  electromagnetic interaction.

\item Hadrons tend to decay to more hadrons.  Only when a decay into
  hadrons is energetically impossible do hadrons decay to photons or
  leptons.

\end{itemize}

\begin{table}[!b]
\caption{Decay times for various processes.}
\label{table:process_decay_times}
\begin{center}
\begin{tabular}[tbp]{lc}
Process & Typical Decay Time  \\[0.5ex]
\hline\hline
Strong             & $10^{-23}$ s \\[0.5ex]
Electromagnetic    & $10^{-20}$ s \\[0.5ex]
Weak $\to$ hadrons & $10^{-10}$ s \\[0.5ex]
Weak $\to$ leptons & $10^{-7}$ s  \\[0.5ex]
\hline
\end{tabular}
\end{center}
\end{table}
%edited to remove confusing third column
%\begin{table}[tbp]
%\caption{Decay times for various processes.}
%\label{table:process_decay_times}
%\begin{center}
%\begin{tabular}[tbp]{lcc}
%Process & Typical Decay Time & Observed Range \\[0.5ex]
%\hline\hline
%Strong             & $10^{-23}$ s & $10^{-22}$--$10^{-24}$ s \\[0.5ex]
%Electromagnetic    & $10^{-20}$ s & $10^{-16}$--$10^{-20}$ s \\[0.5ex]
%Weak $\to$ hadrons & $10^{-10}$ s & $10^{-9} $--$10^{-13}$ s \\[0.5ex]
%Weak $\to$ leptons & $10^{-7}$ s  & $10^{-8} $--$10^0$ s      \\[0.5ex]
%\hline
%\end{tabular}
%\end{center}
%\end{table}

\begin{table}[!b]
\caption{Selected particle masses and lifetimes.}
\label{table:mass_lifetimes}
\begin{center}
\begin{tabular}[tbp]{lcccc}
& Particle & Mass (MeV/$c^2$) & Lifetime & Comments \\
\hline\hline
Baryons & \\
& $p$ & 938.3 & $>10^{30}\units{yrs}$ & \\
& $n$ & 939.6 & 898 s & \\
& $\Lambda$ &  1116 & $2.6\times 10^{-10}$ s & lightest $S = -1$ \\
& $\Sigma^0$ & 1193 & $6\times 10^{-20}$ s & $\to \Lambda + \gamma$ \\
& $\Sigma^-$ & 1197 & $1.5\times 10^{-10}$ s & $\to n + \pi^-$ \\
& $\Sigma^{*-}$ & 1387 & $2\times 10^{-23}$ s & $\to \Lambda + \pi^-$ \\
& $\Omega^-$ & 1672 & $8\times 10^{-11}$ s & lightest $S = -3$\\
Mesons & \\
& $\pi^\pm$  & 139.6 & $2.6\times 10^{-8}$ s & $\to \mu + \nu$\\
& $\pi^0$    & 135   & $0.8\times 10^{-16}$ s & $\to\gamma + \gamma$\\
& $K^\pm$    & 494   & $1.2\times 10^{-8}$ s & lightest $S = \pm 1$\\
\hline
\end{tabular}
\end{center}
\end{table}

\section{Grand Unified Theories (GUTs)}
\label{sec:gut_theories}
Unification of the electrical and magnetic forces was explained in the
late 1800's by Maxwell's comprehensive theory of electricity and
magnetism, which showed that the two forces are really just different
forms of the same fundamental force.  More recently, the discovery of
the $W$ and $Z$ messenger particles (in 1983) verified the {\em Standard
Model} that unifies electromagnetic forces with weak forces (now
referred to in one word as ``electroweak'' forces).  But Einstein's
dream of unification of the fundamental forces is not yet complete,
since strong forces and gravitational forces have not yet been
included in the model.  However, attempts are being made to include
the strong force with the electroweak.

To see how this inclusion would be possible, we must consider one more
type of interaction, one that could change quarks into leptons and
vice versa.  This interaction is one of the key ideas of {\em grand
unified theories}.  But first let us reflect on the striking parallel
between the set of quarks and leptons.  We expand the lepton family to
include quarks and call the resulting group a generation.  See
Table~\ref{table:particle_generations}.

Only the first generation is really ``needed.''  All the normal
everyday material of nuclei, atoms, and objects are made from these.
The other generations are extra pieces of the puzzle of physics.  When
grouped this way, one might ask if there isn't some scheme that unites
the quarks and leptons.  In fact there is.  This is the job of grand
unified theories (GUTs).

\begin{table}[tbp]
\caption{Generations of particles}
\label{table:particle_generations}
\begin{center}
\begin{tabular}[tbp]{lcccc}
 & Leptons & \multicolumn{3}{c}{Quarks} \\
 & & red & blue & green \\
\hline\hline
1st generation & $e^-$ & $u$ & $u$ & $u$ \\
               & $\nu_e$ & $d$ & $d$ & $d$ \\[0.5ex]
2nd generation & $\mu^-$ & $s$ & $s$ & $s$ \\
               & $\nu_\mu$ & $c$ & $c$ & $c$ \\[0.5ex]
3rd generation & $\tau^-$ & $b$ & $b$ & $b$ \\
               & $\nu_\tau$ & $t$ & $t$ & $t$ \\[0.5ex]
\hline
\end{tabular}
\end{center}
\end{table}

The term {\em grand unification} refers to the attempt to unify, in a
single theory, the strong, electromagnetic, and weak interactions.
That there is any hope for this to succeed comes from the fact that
the strengths of the interactions seem to approach each other at short
distances and thus at high energies (recall that to explore ever
smaller-sized details requires more and more energetic probes).  We
saw in the previous chapters that camouflage effects tend to make the
effective color charge (and therefore the interaction strength of the
strong force) {\em smaller} at closer distances.  On the other hand, the
weak and electromagnetic interactions get {\em stronger} at closer
distances.  In fact, the theories describing these interactions are
sufficiently well understood that the distance (and thus the energy)
at which the interaction strengths become comparable can be estimated
to within a factor of 10.  See Fig.~\ref{fig:GUT}.

\begin{figure}[tbp]
\begin{center}
\includegraphics[width=8cm]{fundamental_forces_II/GUT}
\caption{Effective strength vs.\ energy for strong, weak, and
  electromagnetic interactions.}
\label{fig:GUT}
\end{center}
\end{figure}

The energy for grand unification, as seen from the figure, is around
$10^{17}\units{MeV}$. This corresponds to the rest energy of the proposed
messenger particle, the $X$-boson or {\em leptoquark}.  The
$X$-boson's job is to mediate interactions in which quarks turn into
leptons and vice-versa.  With its tremendous mass of about $10^{14}$
proton masses, it could never be produced on earth as a real particle.
Even in its transitory virtual state as a messenger, it would have to
borrow so much energy from the surrounding vacuum, that reactions
involving the $X$ (under present-day earth-like conditions at least)
are expected to be quite unlikely.  However, this boson would make
possible the startling reaction shown in Fig.~\ref{fig:proton_decay}.

\begin{figure}[tbp]
\begin{minipage}{7.5cm}
\caption{Proton decay, mediated by the $X$-boson.}
\label{fig:proton_decay}
\end{minipage}
%\hfill
\begin{minipage}{3cm}
\includegraphics[width=2.5cm]{fundamental_forces_II/proton_decay}
\end{minipage}
\end{figure}

If $X$-bosons are real, the reaction pictured --- proton decay ---
would become possible, although extremely rare.  According to present
theories, the mean lifetime of the proton would be greater than $10^{30}$
years.  Even though this is 20 orders of magnitude longer than the
lifetime of the universe, it may still be testable here on earth (see
Problem~
%\ref{chapter:interactions}.
\ref{prob:proton_decay}).

\section{Universal Symmetry}
\label{sec:universal_symmetry}

One of the goals of grand unified theories is to show that the strong,
weak, and electromagnetic interactions are really different aspects of
a single type of universal interaction.  As we've just seen, by
studying how the strengths of these interactions vary with energy and
distance, one learns that at high enough energies, the interaction
strengths would become equal (see Fig.~\ref{fig:GUT}).  At this
energy, one could observe directly what is known as universal
symmetry.  All the interactions would be on the same footing, look the
same, and have the same strength.

At this grand unification energy, all interactions would be equally
possible, including ones mediated by the $X$-boson, mentioned above in
proton decay.  In order to really see the $X$-boson in action, we
would need to have energies available comparable to the $X$-boson's
rest energy $\sim 10^{17}\units{MeV}$.  Recall that the world's biggest
accelerator, the {\it Large Hadron Collider} in Europe, runs at a 
paltry $10^7\units{MeV}$.  A modern linear
accelerator of $10^{17}\units{MeV}$ would have to extend beyond the outer
reaches of Pluto's orbit around the sun.  So how can we hope to study
such high energies where universal symmetry would be apparent?

The answer is during the early universe!  Shortly after the Big Bang,
the average thermal energies of particles were well in excess of the
grand unification scale.  In those early moments, reactions occurred
that determined the character of our present universe.  Since we
believe that the same laws of physics governed then as now, we can use
the early universe as the testing ground for modern particle physics.
The details of the physics of the early universe are the subject of
the next chapter.


\newpage

\section*{Problems}
\label{sec:interactions_problems}
\markright{PROBLEMS}

\begin{problem}
Construct a reaction diagram for each of the following.
  \begin{enumerate}
  \item $\Delta^{++} \to p + \pi^+$
  \item $p + e^- \to n + \nu_e$
  \end{enumerate}
\label{prob:reaction_diagram_i}
\end{problem}

\begin{problem}
Construct a reaction diagram for each of the following:
  \begin{enumerate}
  \item $\pi^- + \Lambda \to K^- + n$
  \item $K^- \to \mu^- + \overline\nu_\mu$
  \end{enumerate}
\label{prob:reaction_diagram_ii}
\end{problem}

\begin{problem}
Suppose the proton's mean lifetime is
$10^{31}$ years.  Now consider a pool of water $10\units{m} \times
10\units{m} \times 3\units{m}$.  Estimate the number of protons in
the pool and then the number that you would expect to decay in one
year.  For this calculation, consider only the protons not bound in a
complex nucleus; that is, only the protons in hydrogen atoms. 
\label{prob:proton_decay} 
\end{problem}

\begin{problem}
$\Sigma^-(1197)$ decays into hadrons only.  List all the
possible decay schemes that conserve baryon number, energy, and
charge, using the tables given in Chapter~\ref{chapter:particles} as a
resource. 
\label{prob:Sigma_minus_decay}
\end{problem}

\begin{problem}
For each of the following reactions, tell which interaction is
  involved (strong, weak or electromagnetic).
  \begin{enumerate}
  \item a green up quark emits a gluon
  \item $\gamma \to e^+ + e^-$
  \item $\nu_e + n \to e^- + p$
  \end{enumerate}
\label{prob:identify_interaction}
\end{problem}

\begin{problem}
Which lives longer, $\Sigma^0(1192)$ or $\Lambda(1116)$?  Why?
\label{prob:compare_lifetime}
\end{problem}

\begin{problem}
An $\Omega^-$ particle is constructed from three $s$-quarks and
  is the lightest $S = -3$ baryon.  Refer to
  Table~\ref{table:mass_lifetimes} to answer the following:
  \begin{enumerate}
  \item Why must strange\-ness ($S$) change when $\Omega^-$ decays?
  \item By what interaction does $\Omega^-$ decay?
  \item Why does $\Omega^-$ live relatively long?
  \end{enumerate}
\label{prob:Omega_minus}
\end{problem}

\begin{problem}
The particles $\Xi^-(1322)$ and $\Xi^{*-}(1535)$ both carry
strange\-ness $S = -2$.  Which decays faster?  Why?  When the faster
  one decays, what are the likely decay products?
\label{prob:Xi_minus}
\end{problem}

\begin{problem}
Pions are the lightest hadrons.  What can they decay to?  Why do
  $\pi^+$ and $\pi^-$ live so much longer than $\pi^0$?
\label{prob:pion_decay}
\end{problem}

\begin{problem}
In Section \ref{sec:gut_theories} we discussed the possibility of proton
decay.  Fig.~\ref{fig:proton_decay} shows a {\em hypothetical} process
in which proton decay is mediated by a {\em proposed} $X^{4/3}$ boson,
which carries a $+\textstyle{\frac{4}{3}}$ charge.  (The $X^{4/3}$ boson 
has not been observed, and is not one of the particles included in the 
tables in this supplement.)  Note that the general rule for
interactions involving an $X$-boson would be:
  $$
  \mbox{quark} +\mbox{quark} \to X \to \mbox{antiquark} +
  \mbox{antilepton}
  $$
Another possible proton decay scheme, using a $X^{1/3}$ boson (with
$+\textstyle{\frac{1}{3}}$ charge), would be $p \to \pi^+ + \overline\nu_e$.  Construct
a reaction diagram like Fig.~\ref{fig:proton_decay} showing this 
hypothesized decay process.  Be sure to conserve charges at each vertex!
\label{prob:X_boson}
\end{problem}

\begin{problem}
In Fig.~\ref{fig:neutron_decay}, label the colors of any
particle in the diagram that has a color.  Be sure to satisfy the
{\em colorless rule} for any hadron, and don't label colors on any
particle that does {\em not} carry color.  (Note: there are several
right answers to this question.)
\label{prob:neutron_decay_reaction_diagram}
\end{problem}

\begin{problem}
For each of the following reaction diagrams, identify the
messenger particle and state the type of interaction involved
(strong, weak, electromagnetic, or gravitational).  Denote the
colors and/or charges for each messenger, where appropriate.
\begin{center}
  \includegraphics[width=9cm]{fundamental_forces_II/missing_particles_2}
\end{center}
\label{prob:identify_missing_info}
\end{problem}
\vfill

%%%%%%%%%%%%%%%%%%%%%%%%%%%%%%%%%%%%%%%%%%%%%%%%%%%%%%%%%%%%%%%%%%%

