
%%%%%%%%%%%%%%%%%%%%%%%%%%%%%%%%%%%%%%%%%%%%%%%%%%%%%%%%%%%%%%%%%%%%%%%%

\chapter[Particles and Conservation Laws]{`Elementary' Particles and
Conservation Laws}
\label{chapter:particles}

\section{Introduction}
\label{sec:particles:intro}

One of the deepest and most persistent questions in physics is
``What is the universe made of?''  In this chapter we begin to
address this question.  The answer seems to be {\em particles},
and the harder we bang protons or electrons together, the more new
types of particles appear. In this chapter, we discuss our current
understanding of the fundamental building blocks of nature.
Specifically, we discuss categories for particle types, some of
their properties, and the way particles behave in various
reactions. And most importantly, you'll see how conservation laws
begin to make sense of the jumble of ``elementary'' particles.  In
the next two chapters, we will discuss how these particles
interact with each.

A word about the term {\em elementary particle}: ideally, an elementary
particle has a set of fixed properties, like mass, spin, charge, etc.,
and cannot be further subdivided into smaller constituents.  Atoms
were once considered elementary until they were found to be made of
protons, neutrons, and electrons.  Only a few of the particles
discussed in this chapter are truly elementary.  Thus the use of
quotes.

\section{Particle Types}
\label{sec:particle_types}

It appears that on the fundamental level, all matter consists of
particles. But what are these particles?  How can they be
distinguished and identified?  Are there a few basic types from
which all matter is constructed?  These are some of the questions
to be addressed in this section.

Now to the question of identifying particles.  Basically, a particle
is characterized by its mass and charge, but there are many other
properties that contribute to identification.  In the relativistic
momentum and energy lab from Physics 211, you calculated the mass and
charge of an invisible particle by invoking the conservation laws of
energy and momentum.  You studied the reaction
\begin{equation}
\pi^- + p \to \pi^- + \pi^+ + X
\end{equation}
and determined that particle $X$ had a mass about as big as a proton,
but was charge neutral.  This was sufficient information to identify
$X$ as a neutron.

In the 1930's, the only particles known were protons, neutrons,
electrons, and photons.  As better detectors became available, new
particles were identified from reactions initiated by cosmic rays and
as products of radioactive decay.  For instance, when cosmic rays
interact with gas in the upper atmosphere {\em muons} are created.
These are negatively-charged particles that decay quickly into electrons,
releasing large amounts of energy by the reaction 
$\mu^- \to e^- + \mbox{energy}$.  
Muons are not a part of normal matter and their existence
was a puzzle for physicists.

Another milestone in particle physics was the discovery of the
positron, also known as an antielectron.  In 1932, Carl Anderson
succeeded in identifying a positive particle in cosmic rays with the
same mass as an electron.  Using a cloud chamber in a magnetic field,
he photographed an electron-like track being deflected in the opposite
direction that one would expect from an electron.  This was the first
direct evidence of an {\em antiparticle} and served to confirm Dirac's
theoretical prediction that all particles have antiparticles.  Since
then antiparticles have been found for essentially all known
particles.  Antiparticles have the same mass and spin as their
corresponding particles, but take the opposite sign for additive
properties like charge.

Useful as these studies were, the experimental physicists' dependence
on naturally occurring energetic particles was very restrictive.  In
the 1940's and 50's, particle accelerators were developed that enabled
physicists to extend vastly the range of energies and particles
available for study.  Particles like pions ($\pi$) and neutrinos
($\nu$) were found in reactions such as
\begin{equation}
p + n \to n + n + \pi^+,
\end{equation}
and
\begin{equation}
n \to p + e^- + \nu_e.
\end{equation}

Most of the particle physics of the 1950's and 1960's consisted of banging
particles together with more and more energy and seeing what came out.
With higher energy accelerators, more and more kinetic energy of the
incident particle could be converted to rest energy of new particles.
An explosion of new particles, with names like lambda ($\Lambda$),
sigma ($\Sigma$), kaon ($K$), tau ($\tau$) and omega-minus
($\Omega^-$), appeared on the scene (see Tables~\ref{table:leptons}
through \ref{table:baryons}).  From the three or four particles known in
the first third of the 20$^\text{th}$ century, the list of ``elementary'' 
particles
had grown to several hundred.  Fermi is said to have remarked that if
he had known that there were so many particles whose properties he was
expected to memorize, he would have taken up botany!\footnote{Halliday
and Resnick, Fundamentals of Physics Extended, Third Edition, John
Wiley and Sons (1988), p. 1126.}

A list of particles known today would extend to many hundreds of
entries.  The problem of devising separate symbols to name the
particles long ago exhausted the resources of the Greek and Latin
alphabets and numbering particles with their masses is now
commonplace.  A database maintained by the \textit{Particle Data
Group} (see \verb+http://pdg.lbl.gov/+)  contains entries 
like $\Delta$(1910) or $f'$(1523), with a listing of the particle's 
properties (mass, charge, spin, lifetime,
etc.).  But to make any sense of this huge list of particles and
properties, we must look at the classification schemes available.

\section{Classification of Particles}
\label{sec:particle_classification}

Historically, particles were classified by their masses.  Two
particles with exactly the same mass are almost surely identical,
unless some other obvious difference (like different charges)
turns up. Even then, physicists give the particles the same symbol
(like $e^-$ for electrons, $e^+$ for positrons) and look for some
underlying symmetry that causes them to have the same mass.

The first rough classification assigned the labels {\em leptons}, {\em
mesons}, and {\em baryons} to particles, names derived from the Greek
words for light, medium, and heavy.  Thus low mass particles like
electrons were called leptons, middle mass particles like pions were
called mesons, and heavy particles (protons and neutrons) were called
baryons.

As more particles entered the scene, a scheme based just on mass
became inadequate.  For instance, a muon has a mass close to that
of a pion, but behaves much more like an electron.  The present
classification of particles is now based on two factors: the type
of force the particle ``feels,''\footnote{We will be discussing
the fundamental forces/interactions in the next chapter.  For the
moment, though, you should be aware that there are four
fundamental forces: the \textit{strong} force, the
\textit{electromagnetic} force, the \textit{weak} force and the
\textit{gravitational} force.} and the spin of the particle. Apparently
all particles feel the weak interaction, but only some can feel
the strong interaction. The constituents of matter that feel the
strong interaction are called {\em hadrons}; those that do not are
{\em leptons}.  (Messenger particles,\footnote{We will be
discussing messenger particles in the next chapter as well.} like
photons and $W$'s, are neither; they are not constituents of
matter but force carriers.) As for a classification based on spin,
those particles with integral spin are called {\em bosons}, those
with half-integral spin are called {\em fermions}, as we have
seen. Let's now look at the present definitions of leptons,
mesons, and baryons.

{\bf Leptons.}  Leptons are those particles that do not participate in
the strong interaction.  All leptons have spin-$\textstyle{\frac{1}{2}}$ 
and are thus
fermions.  Their masses range from 0 to $1.78\units{GeV/$c^2$}$.  Leptons
appear to have no internal structure (down to $10^{-18}\units{m}$ at least)
and thus appear to be true elementary particles.  There are only a
handful of different kinds and they come in just three groups, or
families, as seen in Table~\ref{table:leptons}.  Some examples are
electrons, neutrinos, and muons.

\begin{table}
\caption{The leptons}
\label{table:leptons}
\begin{tabular}[b]{llclrrrrc}
&& Mass && \multicolumn{4}{|c|}{Additive}& Anti- \\
\multicolumn{2}{l}{Symbol/Name} & (MeV/$c^2$) & Spin\hspace{4mm} &
\multicolumn{1}{|r}{$Q$}
 & $L_e$ & $L_\mu$ & \multicolumn{1}{r|}{$L_\tau$} & particle \\
\hline\hline
$e^-$   & electron          & 0.511    & 1/2 & $-1$ & 1 & 0 & 0 & $e^+$      \\
$\nu_e$ & electron-neutrino & $\sim 0$ & 1/2 & $ 0$ & 1 & 0 & 0 & $\overline\nu_e$\\
[0.5ex]
$\mu^-$  & muon             & 106      & 1/2 & $-1$ & 0 & 1 & 0  & $\mu^+$ \\
$\nu_\mu$ & mu-neutrino     & $\sim 0$ & 1/2 & $ 0$ & 0 & 1 & 0 & $\overline\nu_\mu$\\
[0.5ex]
$\tau^-$ & tau              & 1777     & 1/2 & $-1$ & 0 & 0 & 1  & $\tau^+$ \\
$\nu_\tau$ & tau-neutrino   & $\sim 0$ & 1/2 & $ 0$ & 0 & 0 & 1 & $\overline\nu_\tau$\\
\hline
\end{tabular}\\[0.5ex]
Notes:
%\vspace{-4mm}
\begin{itemize}
%\setlength{\itemsep}{-2mm}
\item All leptons have $B=0$ and $S=0$.
\item For antiparticles, the signs of the additive properties are reversed.
\end{itemize}
\end{table}

{\bf Mesons.}  Mesons are those particles that participate in the
strong interaction and have integer spin.  That is, those hadrons that
are also bosons are called mesons.  Their masses range from 
$0.135\units{GeV/$c^2$}$ up to $11\units{GeV/$c^2$}$ or more, and more are
still being discovered.  Only a few have lifetimes longer than about
$10^{-20}\units{s}$.  Unlike leptons or baryons the number of mesons
is not conserved --- mesons can be created or destroyed in reactions.
Pions and kaons are two examples of meson types.  Table~\ref{table:mesons}
lists several others.

\begin{table}[tbph]
\caption{Selected mesons}
\label{table:mesons}
\begin{tabular}[tbp]{llclrcrc}
&& Mass && \multicolumn{3}{|c|}{Additive}& \\
\multicolumn{2}{l}{Symbol/Name} & (MeV/$c^2$) & Spin\hspace{4mm} & \multicolumn{1}{|r}{$Q$}
 & $B$ & \multicolumn{1}{r|}{$S$} & Antiparticle \\
\hline\hline
$\pi^0$  & pion      & 135   & 0 & $ 0$ & 0 & $0$  & $\pi^0$ \\[0.5ex]
$\pi^+$  & pion      & 140   & 0 & $+1$ & 0 & $0$  & $\pi^-$ \\
$K^+$    & kaon      & 494   & 0 & $+1$ & 0 & $+1$ & $K^-$   \\
$K^0$    & kaon      & 498   & 0 & $ 0$ & 0 & $+1$ & $\overline K^0$\\[0.5ex]
$\eta$   & eta       & 548   & 0 & $ 0$ & 0 & $ 0$ & $\eta$  \\
$\eta'$  & eta-prime & 958   & 0 & $ 0$ & 0 & $ 0$ & $\eta'$ \\[0.5ex]
$\rho^+$ & rho       & 775   & 1 & $+1$ & 0 & $ 0$ & $\rho^-$\\
$\rho^0$ & rho       & 775   & 1 & $ 0$ & 0 & $ 0$ & $\rho^0$\\
$\omega$ & omega     & 783   & 1 & $ 0$ & 0 & $ 0$ & $\omega$\\
\hline
\end{tabular}\\[0.5ex]
Notes:
%\vspace{-4mm}
\begin{itemize}
%\setlength{\itemsep}{-2mm}
\item All mesons have $L_e=L_\mu=L_\tau = 0$.
\item For antiparticles, the signs of the additive properties are reversed.
\item Some mesons are their own antiparticles.
\end{itemize}
\end{table}

{\bf Baryons.}  Baryons are those particles that participate in the
strong interaction and have half-integer spin.  That is, hadrons that
are also fermions are called baryons.  As a class, they are the heaviest
and most numerous type of particle with hundreds of members and masses
ranging from $0.938\units{GeV/$c^2$}$ up to $2.6\units{GeV/$c^2$}$.
Both mesons and baryons seem to have a size and internal structure
on the scale of $10^{-15}$--$10^{-16}\units{m}$.  Baryons are
``proton-like,'' in that all baryons eventually decay into protons,
the lightest baryon.  Some other baryons are the neutron, the lambda
($\Lambda$) and the omega minus ($\Omega^-$).  A few baryons are listed
in Table~\ref{table:baryons}.

\begin{table}[htbp]
\caption{Selected baryons}
\label{table:baryons}
\begin{tabular}[tbp]{llclrcr}
&& Mass && \multicolumn{3}{|c}{Additive} \\
\multicolumn{2}{l}{Symbol/Name} & (MeV/$c^2$) & Spin\hspace{4mm} &
\multicolumn{1}{|r}{$Q$}
 & $B$ & $S$ \\
\hline\hline
$p$           & proton       & 938.3 & 1/2 & $+1$ & 1 & $0$  \\
$n$           & neutron      & 939.6 & 1/2 & $ 0$ & 1 & $0$  \\[0.5ex]
$\Lambda$     & lambda       & 1116  & 1/2 & $ 0$ & 1 & $-1$ \\
$\Sigma^+$    & sigma        & 1189  & 1/2 & $+1$ & 1 & $-1$ \\
$\Sigma^0$    & sigma        & 1192  & 1/2 & $ 0$ & 1 & $-1$ \\
$\Sigma^-$    & sigma        & 1197  & 1/2 & $-1$ & 1 & $-1$ \\[0.5ex]
$\Delta^{++}$ & delta        & 1232  & 3/2 & $+2$ & 1 & $ 0$ \\
$\Delta^+$    & delta        & 1232  & 3/2 & $+1$ & 1 & $ 0$ \\
$\Delta^0$    & delta        & 1232  & 3/2 & $ 0$ & 1 & $ 0$ \\
$\Delta^-$    & delta        & 1232  & 3/2 & $-1$ & 1 & $ 0$ \\[0.5ex]
$\Xi^0$       & cascade      & 1315  & 1/2 & $ 0$ & 1 & $-2$ \\
$\Xi^-$       & cascade      & 1322  & 1/2 & $-1$ & 1 & $-2$ \\[0.5ex]
$\Sigma^{*+}$ & sigma-star   & 1383  & 3/2 & $+1$ & 1 & $-1$ \\
$\Sigma^{*0}$ & sigma-star   & 1384  & 3/2 & $ 0$ & 1 & $-1$ \\
$\Sigma^{*-}$ & sigma-star   & 1387  & 3/2 & $-1$ & 1 & $-1$ \\[0.5ex]
$\Xi^{*0}$    & cascade-star & 1532  & 3/2 & $ 0$ & 1 & $-2$ \\
$\Xi^{*-}$    & cascade-star & 1535  & 3/2 & $-1$ & 1 & $-2$ \\
$\Omega^-$    & omega-minus  & 1672  & 3/2 & $-1$ & 1 & $-3$ \\
\hline
\end{tabular}\\[0.5ex]
Notes:
%\vspace{-4mm}
\begin{itemize}
%\setlength{\itemsep}{-2mm}
\item All baryons have $L_e = L_\mu = L_\tau = 0$.
\item An antiparticle is indicated with a bar, e.g.,
$\overline p$ is an antiproton.
\item For antiparticles, the signs of the additive properties are reversed.
\end{itemize}
\end{table}

\section{Conservation Laws}
\label{sec:conservation_laws}

Let's now try to analyze a simple reaction like you might see in a
bubble chamber, in which part of the incoming energy is used to create
a new particle, say a pion.
\begin{equation}
p + p \to p + p + \pi
\label{eq:create_pion}
\end{equation}
To analyze this reaction, we use conservation laws.  Three
conservation laws of particle physics have analogs in classical
physics: conservation of charge, conservation of energy, and
conservation of angular momentum.  First use the law of
conservation of charge, which states that in any reaction the
total charge before and after the reaction must be equal.  Protons
carry a single unit of positive charge, but pions come in three
varieties: $\pi^+$, $\pi^0$, and $\pi^-$.  We can thus use
conservation of charge to determine which variety of pion appears
in the reaction Eq.~(\ref{eq:create_pion}).  The answer is the
neutral pion, $\pi^0$.

Next, let's see how to apply the conservation of energy.  Here is a
caution about applying energy conservation to reactions in which {\em
two or more} particles are present initially.  Unless you are told
about the incident particles' kinetic energies, energy conservation
alone cannot rule out any reaction.  But in a decay, when only a {\em
single} particle is present before the reaction, energy conservation
requires that the decay products be less massive than the original
particle.  In a decay, a single unstable particle disappears to be
replaced by two or more less massive particles.  (Can you see why at
least two particles are needed?)

Here are some typical decays.  A positive pion can decay into an
antimuon ($\mu^+$) and a mu-neutrino ($\nu_\mu$) with a mean life of
$2.6 \times 10^{-8}\units{s}$:
\begin{equation}
\pi^+ \to \mu^+ + \nu_\mu
\label{eq:pion_decay}
\end{equation}
The muon is also unstable and decays on the average after $2.2 \times
10^{-6}\units{s}$:
\begin{equation}
\mu^+ \to e^+ + \nu_e + \overline\nu_\mu
\label{eq:muon_decay}
\end{equation}

Finally let's see how to apply angular momentum conservation.  In
particle physics, angular momentum is quite complicated, since it is a
quantized vector quantity that includes contributions from both spin
and orbital angular momentum.  A complete description of this
conservation law would thus require techniques beyond the scope of
this course.  However, we can derive one simple rule, easy to apply,
based on the fact that an even number of half-integral spins always
combine to give integral spin.

\boxittext{
{\bf The Fermion Rule for angular momentum conservation:} A
reaction can only occur if the number of fermions before the reaction
and the number after are either both odd or both even.}

These three conservation laws --- charge, energy, and angular
momentum --- are absolute: they apply in all known reactions.

In addition to the absolute conservation laws of charge, energy, and
angular momentum, there are two other well-tested conservation laws,
those for lepton number and baryon number.  Although some modern field
theories predict their violation in reactions like proton decay, these
violations have never been conclusively observed.  So until we discuss
{\em grand unified theories} in Chapter~\ref{chapter:interactions}, we will
assume that lepton and baryon numbers are conserved.

So what is lepton number?  Actually it's quite simple.  Each lepton is
assigned a lepton number of $L = +1$.  The antiparticles of leptons
have $L = -1$.  All other particles have $L = 0$.  Now consider a
reaction like pion decay, Eq.~(\ref{eq:pion_decay}).  The pion is not
a lepton, so it has $L = 0$.  The $\mu^+$ is the antiparticle of the
$\mu^-$, a lepton, so $\mu^+$ has $L = -1$.  Finally, a neutrino is a
lepton, with $L = 1$.  Lepton number is an additive quantity, so
simple addition shows that both sides total $L = 0$, and the decay is
allowed.

Actually, conservation of lepton number is slightly more complicated
than this simple example implies.  Consider, for example, the following
muon decay
\begin{equation}
\mu^+ \to e^+ + \nu_? + \overline\nu_?,
\label{eq:muon_decay_q}
\end{equation}
 and a possible alternative decay
\begin{equation}
\mu^+ \to e^+ + \gamma.
\label{eq:not_muon_decay}
\end{equation}
The second decay seems simpler and also appears to conserve lepton
number, but in fact, a decay like Eq.~(\ref{eq:not_muon_decay}) has
never been observed.  A situation like this is a puzzle for
physicists.  Why doesn't Eq.~(\ref{eq:not_muon_decay}) ever occur?
There must be something new going on that prevents it.  This
``something new'' is the existence of a previously undiscovered
conservation law!

The new law is this:  Each of the three types of leptons is conserved
separately.  Thus, there are electron neutrinos ($\nu_e$), muon
neutrinos ($\nu_\mu$), and tau neutrinos ($\nu_\tau$).  The electron
and its neutrino each carry $L_e = +1$, $L_\mu = L_\tau = 0$, and
similarly for the other leptons.  Thus, the muon decay of
Eq.~(\ref{eq:muon_decay}) is actually
\begin{equation}
\mu^+ \to e^+ + \nu_e + \overline\nu_\mu
\end{equation}
and the candidate decay Eq.~(\ref{eq:not_muon_decay}) is ruled out
because neither $L_e$ nor $L_\mu$ is conserved.
So when we say ``conserve lepton number," it really means
``conserve all three types of lepton number:  $L_e$, $L_\mu$, and 
$L_\tau$.'' {\bf REMEMBER: There are THREE separate lepton numbers, and 
each needs to be conserved separately.} E.g., if $L_e = 1$ and $L_\mu =
L_\tau = 0$ before a reaction, and $L_e = 0, L_\mu = 1,$ and $L_\tau = 0$ after
the reaction, then this violates lepton number conservation since $L_e$ isn't
conserved and $L_\mu$ isn't conserved.

Similar observations lead to the formulation of the law of
conservation of baryon number.  Consider the two candidate reactions
below:
\begin{equation}
\pi^0 \to \gamma + \gamma
\label{eq:baryon_allowed}
\end{equation}
and
\begin{equation}
n + n \to \gamma + \gamma
\label{eq:not_baryon_allowed}
\end{equation}
Both are allowed by all the conservation laws described so far, but
the first occurs all the time, while the second never occurs.  Why
not?  A new conservation law must be operating: conservation of baryon
number.  Neutrons are baryons, so they each have $B = 1$.  Photons and
pions are not baryons so $B = 0$ for them.  Any antiparticles of
baryons have $B = -1$.  Finally, since baryon number is additive, we
have that the decay in Eq.~(\ref{eq:baryon_allowed}) is allowed
because $0=0$, while the reaction in Eq.~(\ref{eq:not_baryon_allowed})
is forbidden because $2 \neq 0$.

These sorts of {\em ad hoc} conservation laws seem to be begging the
question.  Later on we'll encounter a deeper reason for these laws,
especially the law of conservation of baryon number.


\section{Strangeness}
\label{sec:strangeness}

We now come to the first conservation law that is partially violated:
conservation of strange\-ness.  Strange\-ness is a property carried
only by hadrons, and it is conserved during strong and electromagnetic
interactions, but not necessarily conserved in weak interactions.
Its name arises from the strange behavior observed in the production and
decays of certain hadrons.  Consider the following sequence, a result
of bombarding stationary protons with high energy pions:
\begin{equation}
\pi^- + p \to \Lambda + K^0,
\label{eq:lambda_k_production}
\end{equation}
followed by
\begin{equation}
\Lambda \to p + \pi^- \qquad \mbox{lifetime} \sim 3\times 10^{-10}\units{s}
\label{eq:lambda_decay}
\end{equation}
and
\begin{equation}
K^0 \to \pi^+ + \pi^- \qquad \mbox{lifetime} \sim 10^{-9}\units{s}.
\label{eq:k_decay}
\end{equation}
Particles like $\Lambda$ and $K$ always seem to be produced in pairs.
One never sees either of the following candidate reactions.
\begin{eqnarray}
\pi^- + p &\to& n + K^0 \nonumber\\
\pi^- + p &\to& \Lambda + \pi^0.
\label{eq:strange_violation}
\end{eqnarray}

Let's analyze these reactions.  Since the $\Lambda$ eventually decays
to a proton, it must be a baryon, while $K$ decays only to pions,
marking the kaon as a meson.  Then the reactions in both
Eq.~(\ref{eq:lambda_k_production}) and
Eq.~(\ref{eq:strange_violation}) conserve charge, spin, and baryon
number, but Eq.~(\ref{eq:strange_violation}) never occurs.  Why not?
The answer once again is that a new conservation law is being
violated.  Lambdas and kaons must carry a new property (in equal but
opposite amounts) that pions and protons don't have.  This new
property, which particle physicists of the 1960's named {\em
strange\-ness}, comes in integer amounts.  The kaon, by convention, has
$S = +1$ and the lambda has $S = -1$.  Strange\-ness then balances in
Eq.~(\ref{eq:lambda_k_production}) but not in
Eq.~(\ref{eq:strange_violation}).  This, then, would ``explain'' why
the reactions in Eq.~(\ref{eq:strange_violation}) are never observed.

But what about the decays of Eq.~(\ref{eq:lambda_decay}) and
Eq.~(\ref{eq:k_decay})?  Don't they violate strange\-ness
conservation?  The answer is yes, they do.  But these decays proceed
by the weak interaction.  The clue that these are weak decays is the
particles' relatively long lifetimes, a topic we discuss further in
Chapter~\ref{chapter:interactions}.  Suffice it to say that $\Lambda$
and $K^0$ are produced together by the strong interaction, conserving
strange\-ness, but they decay separately by the weak interaction,
which does not conserve strange\-ness.

\section{The Eightfold Way and Quarks}
\label{sec:eightfold_way}

In the 1960's, newer and bigger accelerators came on line, and the
proliferation of newly created particles continued.  There are now
literally hundreds of so-called elementary particles.  A
classification scheme was needed, and in 1962 Murray Gell-Mann and
Yuval Ne'eman proposed the Eightfold Way.  Shortly thereafter,
Gell-Mann suggested that the various hadrons were actually
composed of yet smaller entities called quarks.

Gell-Mann and Ne'eman studied the relationships among various
groups of particles.  They were looking especially for ways to
organize the hadrons into groups with common dynamical properties.
For instance, particles that differ only in charge (and slightly
in mass) are already given the same symbol (like $\Sigma^+$,
$\Sigma^0$, and $\Sigma^-$).  Experimental evidence indicates that
these particles behave identically during strong interactions.
There is also good evidence from nuclear physics that the strong
interaction does not depend on a particle's charge.  Another way
to describe this independence is to say that the strong
interaction possesses charge symmetry.

Another common dynamical property among hadrons seems to be
strange\-ness.  Particles differing only in strange\-ness (and
slightly in mass) are observed to behave identically during strong
interactions. Thus, the strong interaction also possesses
strange\-ness symmetry.

Let's see how Gell-Mann and Ne'eman used these symmetry ideas to
group the hadrons.  The classification scheme they devised, called
the Eightfold Way, grouped together hadrons having the same spin
and baryon number but different charge and strange\-ness.  Because
of charge and strange\-ness symmetry, the particles within one
group should be interchangeable with respect to their behavior in
strong interactions and have masses in a fairly small range.

\begin{figure}[tbp]
\begin{minipage}[t]{6cm}
\begin{center}
\includegraphics[width=5cm]{particles/baryon_SvsQ}
\caption{Strange\-ness vs.\ charge plot for the eight spin 1/2
baryons.} \label{fig:baryon_SvsQ}
\end{center}
\end{minipage}
\hfill
\begin{minipage}[t]{6cm}
\begin{center}
\includegraphics[width=5cm]{particles/meson_SvsQ}
\caption{The nine spin zero mesons on a strange\-ness vs. charge
plot.} \label{fig:meson_SvsQ}
\end{center}
\end{minipage}
\end{figure}

As an example, let's look at the eight known spin 1/2 baryons.
The upper part of Table~\ref{table:baryons} lists their
properties.  They all have the same spin and baryon number, and
their masses are in a fairly small range (about 0.940 to
$1.340\units{GeV/$c^2$}$).  They do differ in charge and strange\-ness
however.  When one plots the particles on a strange\-ness versus
charge plot, a distinctive hexagonal pattern emerges.  See
Fig.~\ref{fig:baryon_SvsQ}.

Gell-Mann and Ne'eman noticed that this hexagonal pattern is the
same as that which arises in the mathematical {\em group} of
$3\times 3$ matrices called $\mathrm{SU}(3)$.  An important property of
these matrices is that they can all be expressed in terms of eight
special basis matrices.  Although a detailed explanation of the
theory of groups is far beyond the scope of this text, some
development of the correspondence between particle plots and
matrices is in order.

If we look at the strange\-ness versus charge plot, we find that
there are certain motions that go from one particle to another.
One can move along any of the three directions parallel to the
sides of the hexagon.  The two horizontal motions (left or right)
correspond to the symmetry that increases or decreases the charge
by one unit.  The two vertical motions (up or down) correspond to
increasing or decreasing the strange\-ness by one unit.  Finally,
the diagonal motions will change both charge and strange\-ness by
one unit.  These six operations, plus one for measuring a
particle's charge and another for measuring strange\-ness,
correspond to the eight basis matrices for the group of $3\times
3$ matrices, and lead to the name Eightfold Way. The Eightfold Way
is also a term from Eastern philosophy and refers to the steps
along the Buddhist path to enlightenment.  But don't read too much
into the name: Gell-Mann was just having fun with it.


Let's see how the Eightfold Way gives some structure to a
different set of particles, the nine known spin zero mesons.
Again, locating the particles on a strange\-ness versus charge
plot shows the hexagonal pattern.  The operations that change or
measure charge and strange\-ness are similar to those for the
baryons.

It turns out that all the known particles fit into similar
patterns. In fact, Gell-Mann was so convinced that his patterns
were correct that, when he found a hole in the spin 3/2 baryons'
pattern, he predicted the existence of a new particle and
described its properties.  Within weeks of his 1962 prediction,
the $\Omega^-$ particle, with the correct properties, was
discovered in bubble chamber photographs at Brookhaven National
Laboratory, providing strong experimental confirmation of
Gell-Mann's theory.

\section{Quarks}
\label{sec:quarks}

Nice as these patterns of the Eightfold Way are, many questions
remain.  For instance, why are there two particles at the center
of the baryon hexagon of Fig.~\ref{fig:baryon_SvsQ}, and three in
the center of the meson pattern in Fig.~\ref{fig:meson_SvsQ}?  Why
don't other patterns occur?

The fact that the elementary particles fall into neat patterns is
reminiscent of the construction of the periodic table of elements.
There it was found that when elements were arranged in
(approximate) order of increasing mass number, certain groups of
elements (the columns of the table) emerged with similar chemical
properties.  Once the periodic table was well established, the
jump was quickly made to understanding atomic structure, with
electrons in shells, subshells, and orbitals.

This idea that regular patterns of particle properties could
indicate a regular underlying structure occurred to Gell-Mann, and
in 1964 he proposed the {\em quark model} of elementary particle
structure. Gell-Mann's quark model states that all hadrons are
composed of a small number of constituent particles called quarks.
The quarks of Gell-Mann's model come in three different varieties
(or flavors) called up, down, and strange.  The quarks' properties
are summarized in Table~\ref{table:quarks}.  The vast variety of
hadrons comes about as different combinations and configurations
of the quarks, just as the hundreds of different isotopes of atoms
arise from different combinations of protons, neutrons, and
electrons.  

The mass of a hadron is not simply the sum of the masses of its
constituent quarks -- the energies of the bound quarks can affect
the mass of the hadron.\footnote{Don't forget the equivalence of
energy and mass from relativity.} Most of our discussion will focus on quarks combined
together in ground state configurations.  But it is possible for
the same set of quarks to be configured into higher energy states
(i.e., higher mass particles), which usually decay quickly.
(Since Gell-Mann's original model, three more flavors of quarks
have been identified, but we will not study them in detail.)

Let's see how different combinations of quarks can be assembled to
make various particles.  Consider a neutron, a neutral,
non-strange baryon.  Whenever we combine quarks to make a
particle, the particle's additive properties ($Q$, $B$, and $S$)
are simply the sum of the constituent quarks' properties.  Since
quarks have $B = 1/3$, we need at least three quarks to make a
baryon.  let's try to make a neutron from just three.  If we use a
strange quark, the resulting particle would carry strange\-ness.
So to make a neutron, we should use only ups and downs.  But how
many of each?  We can find out by making the charge come out to
zero: the result is one up and two downs.  Thus, in symbols, $n =
(udd)$.

\begin{table}[tbp]
\begin{center}
\caption{Quark properties} \label{table:quarks}
\begin{tabular}[tbp]{cccrrr}
Flavor & Name & Spin & $Q$ & $B$ & $S$ \\
\hline\hline
$u$ & up      & 1/2 & $+2/3$ & $1/3$ & $ 0$ \\[0.5ex]
$d$ & down    & 1/2 & $-1/3$ & $1/3$ & $ 0$ \\[0.5ex]
$s$ & strange & 1/2 & $-1/3$ & $1/3$ & $-1$ \\
\hline
\end{tabular}\\[0.5ex]
\end{center}
\textbf{Note:} the antiquarks ($\overline u$, $\overline d$, $\overline s$)
have the same spin but the opposite $Q$, $B$, and $S$.
\end{table}

To make a meson is even simpler.  Since the baryon number must
come out zero, a quark plus an antiquark will do it.  Let's make a
positive pion, a non-strange meson.  Since $S= 0$, we could try
$\pi^+ =(s\overline s)$, but then the charge would be $-\frac{1}{3}
+ \frac{1}{3}$, which isn't right.  Thus pions are constructed
using just {\em ups} and {\em downs}.

We can summarize our findings about constructing hadrons out of
quarks with the following:

%\parbox{\hsize}{
%\begin{quote}
\boxittext{
\begin{description}
\item[baryons] contain three quarks ($qqq$) \item[antibaryons]
contain three antiquarks ($\overline q\overline
  q\overline q$)
\item[mesons] contain a quark-antiquark ($q\overline q$)
\end{description}
%\end{quote}
%}
}

The quark model is especially convincing because it explains so
well the Eightfold Way patterns of particles.  Consider again the
spin zero mesons.  Figure~\ref{fig:meson_SvsQ} showed a hexagonal
pattern with three particles in the center.  When we make a
similar plot of the nine possible quark-antiquark pairs, we get
the pattern shown in Fig.~\ref{fig:quarkpair_SvsQ}.  There are
nine combinations in exactly the same places on the plot as in
Fig.~\ref{fig:meson_SvsQ}.  A similar situation occurs when we
make three-quark combinations for baryons.  In fact, every
possible combination of up, down, and strange quarks has been
observed as a particle, and every known hadron can be built of
quarks.

\begin{figure}[!bp]
\begin{center}
\includegraphics[width=4.5cm]{particles/quarkpair_SvsQ}
\caption{The nine quark-antiquark combinations on a strange\-ness
  vs. charge plot.  Note the striking similarity to
  Fig.~\protect\ref{fig:meson_SvsQ}.}
\label{fig:quarkpair_SvsQ}
\end{center}
\end{figure}

Once one accepts the idea of quarks, the strong interaction
conservation laws become almost automatic.  Consider the reaction
\begin{equation}
\pi^- + p \to \Lambda + K^0
\end{equation}
By now you could easily check that angular momentum, charge,
baryon number, and strange\-ness are all conserved.  But write the
reaction in terms of quarks:
\begin{equation}
(d+\overline u) + (u+u+d) \to (d+u+s) + (\overline s+d)
\label{eq:quark_reaction}
\end{equation}
Then simply note that quark conservation {\em implies}
conservation of all the additive physical quantities, since each
quark retains those properties regardless of configuration.  By
quark conservation, we mean that a quark can be created or
destroyed only if its antiquark is created or destroyed
simultaneously, as occurs with a $u\overline u$ pair and the
$s\overline s$ pair in Eq.~(\ref{eq:quark_reaction}).  You'll see
this quark description of reactions more vividly when you learn
about reaction diagrams in Chapter~\ref{chapter:interactions}.

\begin{example}{Quark content of the $\Delta^{++}$.}
\label{ex:delta}
Construct from quarks a baryon with charge $+2$, the
$\Delta^{++}$.
%\solution
\begin{solution}
Since a baryon contains three quarks,
and only an up quark has a charge of $+2/3$, it would require
three such quarks to obtain a charge of $+2$.  Thus, the
$\Delta^{++}$ must be made of $(uuu)$.
\end{solution}
\end{example}

\newpage

\section*{Problems}
\label{sec:particles:problems}
\markright{PROBLEMS}

\begin{problem}
Baryons and leptons are both fermions.  In what ways are they
  different?
\label{prob:baryons_vs_leptons}
\end{problem}

\begin{problem}
Mesons and baryons are each sensitive to the strong force.  In
  what way are they different?
\label{prob:mesons_vs_baryons}
\end{problem}

\begin{problem}
Test the conservation laws for energy, charge, baryon number,
  lepton number, and strange\-ness %and also the Fermion Rule, 
for each
  of the following proposed reactions.  Which ones are violated?
  \begin{enumerate}
  \item $\mu^- \to \pi^- + \nu_\mu$
  \item $\pi^- + p \to \Lambda + \overline\nu_e$
  \end{enumerate}
\label{prob:conservation_laws_i}
\end{problem}

%\begin{problem}
%Which type of pion appears in Eq.~(\ref{eq:create_pion})?  How
%  do you know?
%\end{problem}

\begin{problem}
Classify each of the following particles as bosons, fermions,
  hadrons, leptons, baryons, mesons and/or messengers.  Use as many of
  these terms as apply.\par\medskip
  (a) photon \qquad \qquad
  (b) $\Lambda$ \qquad \qquad
  (c) $\pi^+$ \qquad \qquad
  (d) $\nu_e$
\label{prob:classify_particles}
\end{problem}

\begin{problem}
A $\rho^0$ particle initially at rest is observed to decay by
  $\rho^0 \to \pi^+ + \pi^-$.  The pions have a mass of 
  $140\units{MeV/$c^2$}$.  The magnitude of the momentum of each 
  of the oppositely directed pions is measured to be 
  $361.5\units{MeV/$c$}$.  Conserve energy
  and momentum to determine the rest energy of the $\rho^0$.  Compare
  to the value given in Table~\ref{table:leptons}.
\label{prob:rho0_rest_energy}
\end{problem}

\begin{problem}
Use the appropriate conservation laws in each of the strong
  reactions below to determine the additive properties of the unknown
  particle $x$.  Then identify the $x$ particle using
  Tables~\ref{table:leptons} through \ref{table:baryons}.
  \begin{enumerate}
  \item $p + p \to p + \Lambda + x$
  \item $e^- + p \to n + \pi^0 + x$
  \item $K^- + p \to \Xi^- + \pi^0 + x$
  \end{enumerate}
\label{prob:identify_particle_x}
\end{problem}

\begin{problem}
Check strange\-ness conservation to determine which of the
  following reactions may proceed by the strong interaction.
  \begin{enumerate}
  \item $K^- + p \to \Lambda + \pi^0$
  \item $\Xi^- \to \Lambda + \pi^-$
  \item $K^+ \to \pi^+ + \pi^0$
  \item $\Delta^{++} \to p + \pi^+$
  \end{enumerate}
\label{prob:strangeness_and_strong_force}
\end{problem}

\begin{problem}
Use the data in Tables~\ref{table:leptons} through 
  \ref{table:baryons} to check that Eq.~(\ref{eq:pion_decay}) and
  Eq.~(\ref{eq:muon_decay}) do not violate the laws of conservation of
  energy, charge, lepton number, and baryon number.
%and angular momentum.
\label{prob:pion_and_muon_decay}
\end{problem}

\begin{problem}
Can an electron decay by disintegrating into two neutrinos?
  Why, or why not?
\label{prob:electron_stability}
\end{problem}

\begin{problem}
A neutron is massive enough to decay by the emission of a proton
  and two neutrinos.  Why does it not do so?
\label{prob:neutron_decay}
\end{problem}

\begin{problem}
Consider the reaction $\pi^- + p \to \Xi^0 + K^0 + K^0$.
  Assuming this reaction occurs via the strong interaction, determine
  the strange\-ness of the $\Xi^0$.  Compare to the value in
  Table~\ref{table:baryons}.
\label{prob:Xi0_strangeness}
\end{problem}

\begin{problem}
What are the quark constituents for each of the following
  particles?  \par  \medskip
  (a)  $\Lambda$ \qquad\qquad (b) $\Xi^-$ \qquad\qquad (c)  $\pi^+$
\label{prob:quark_constituents}
\end{problem}

\begin{problem}
Use the quark model to explain why there is no baryon with
$S = -2$, $Q = +1$.
\label{prob:quark_model}
\end{problem}

\begin{problem}\
Determine the additive quantum numbers ($B$, $Q$, and $S$), and
  thus the identities, of the particles formed from the following
  combinations of quarks.  Then use the tables in
  Chapter~\ref{chapter:particles} to identify each particle.  Note that a
  quark ingredient list does {\em not} always identify a particle
  uniquely.  Sometimes several particles contain the same set of
  quarks.\par\medskip
  (a) $ddu$ \qquad (b) $uus$ \qquad (c) $d\overline s$
  \qquad (d) $\overline u\overline u\overline d$ \qquad (e) $ssd$
  \qquad (f) $u\overline d$
\label{prob:identify_from_quarks}
\end{problem}

\begin{problem}
Consider the ten baryons with spin 3/2 listed in
  Table~\ref{table:baryons}.  Make a strange\-ness vs.\ charge plot
  for these baryons, similar to Fig.~\ref{fig:baryon_SvsQ}.
  Compare your plot with that in Fig.~\ref{fig:baryon_SvsQ}.
\label{prob:strangeness_vs_charge_plot}
\end{problem}

\begin{problem}
Consider the reaction $\Lambda + \pi^- \to \mbox{baryon} +
  \mbox{meson}$.  What possible baryon-meson pairs can be produced?
  Consider only the simplest cases, in which no quarks are created or
  destroyed, but merely rearrange themselves.
\label{prob:reaction_products}
\end{problem}

\newpage
\thispagestyle{empty}
%%%%%%%%%%%%%%%%%%%%%%%%%%%%%%%%%%%%%%%%%%%%%%%%%%%%%%%%%%%%%%%%%%%%%%%

